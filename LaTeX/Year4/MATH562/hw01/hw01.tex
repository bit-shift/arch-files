% Begin the document and set up the style of the document
\documentclass[a4paper,11pt]{article}

% Install the required packages for the document 
\usepackage{enumitem}
\usepackage{amsmath}
\usepackage{amssymb}
\usepackage{verbatim}
\usepackage{mathtools}
\usepackage{tikz}
\usepackage{nicefrac}
\usepackage{bm}
\usepackage{xlop}

\newcommand{\norm}[1]{\left\lVert#1\right\rVert}


% Page and style settings
%\parskip=8pt
\parindent=0pt
% Right margin
\textwidth=6.25in
% Left margin
\oddsidemargin=0pt
\evensidemargin=0pt
% Bottom margin
\textheight=10in
% Top margin
\topmargin=-0.75in
\baselineskip=11pt
% end of page and other style settings

\renewcommand{\familydefault}{\sfdefault}
\usepackage{calrsfs}
\DeclareMathAlphabet{\pazocal}{OMS}{zplm}{m}{n}

\newcommand{\indep}{\mathrel{\text{\scalebox{1.07}{$\perp\mkern-10mu\perp$}}}}
\newcommand{\p}{\mathbb{P}}
\newcommand{\e}{\mathbb{E}}
\newcommand{\ds}{\displaystyle}
\newcommand{\code}{\texttt}
\newcommand{\HRule}{\rule{\linewidth}{0.5mm}} % Defines a new command for the horizontal lines, change thickness here

\newenvironment{nscentre}
 {\parskip=0pt\par\nopagebreak\centering}
 {\par\noindent\ignorespacesafterend}


\usepackage{fullpage}

\usepackage{titlesec} % Used to customize the \section command
\titleformat{\section}{\bf}{}{0em}{}[\titlerule] % Text formatting of sections
\titlespacing*{\section}{0pt}{3pt}{3pt} % Spacing around sections

\begin{document}
\setlength{\abovedisplayskip}{8pt}{%
\setlength{\belowdisplayskip}{8pt}{%


\text{LSA}
\hfill
\text{University of Michigan}

\begin{nscentre}
	\textbf{MATH562: Continuous Optimisation}\\
	\textbf{Homework 1}\\
\end{nscentre}

\text{Name: Keegan Gyoery}
\hfill
\text{UM-ID: 31799451}

\pagenumbering{arabic}
	\begin{enumerate}[leftmargin=*]
		\item \begin{enumerate}[label=\alph*)]
			\item Let $\ds{f(x_1,x_2) = (x_1 - 3)^2 + (x_2 - 3)^2}$. The domain of $\ds{f}$ is the set  $\ds{D = \mathbb{R}^2}$. The range of $\ds{f}$ is the set $\ds{S = \{f(\mathbf{x}) \in \mathbb{R} : f( \mathbf{x} ) \geq 0, \forall \mathbf{x} \in D\}}$. Considering the Gradient $\ds{(\nabla)}$ and Hessian $\ds{(H)}$ of $\ds{f(\mathbf{x})}$, 
				\begin{align*}
					\nabla f(\mathbf{x}) & = 
					\begin{bmatrix}
						2(x_1-3) \\
						2(x_2-3) \\
					\end{bmatrix}, \\
					Hf(\mathbf{x}) & = 
					\begin{bmatrix}
						2 & 0 \\
						0 & 2 \\
					\end{bmatrix}.
				\end{align*}
				Both the Gradient and Hessian of $\ds{f(\mathbf{x})}$ have domain $\ds{D = \mathbb{R}^2}$. The Gradient has range $\ds{S = M_{2,1}(\mathbb{R})}$, the set of all $\ds{2\times 1}$ matrices over the real numbers. The Hessian has range
				\begin{align*} S & = 
					\begin{bmatrix}
						2 & 0 \\
						0 & 2 \\
					\end{bmatrix}.
				\end{align*}

			\item Let $\ds{f(x_1,x_2) = 4x_1^2 + 9x_2^3 - 36}$. The domain of $\ds{f}$ is the set  $\ds{D = \mathbb{R}^2}$. The range of $\ds{f}$ is the set $\ds{S = \mathbb{R}}$. Considering the Gradient $\ds{(\nabla)}$ and Hessian $\ds{(H)}$ of $\ds{f(\mathbf{x})}$, 
				\begin{align*}
					\nabla f(\mathbf{x}) & = 
					\begin{bmatrix}
						8x_1 \\
						27x_2^2 \\
					\end{bmatrix}, \\
					Hf(\mathbf{x}) & = 
					\begin{bmatrix}
						8 & 0 \\
						0 & 54x_2 \\
					\end{bmatrix}.
				\end{align*}
				Both the Gradient and Hessian of $\ds{f(\mathbf{x})}$ have domain $\ds{D = \mathbb{R}^2}$. The Gradient has range 
				\begin{align*} S & = \left\{A \in M_{2,1}(\mathbb{R}) : A =
						\begin{bmatrix}
							a \\
							b \\
						\end{bmatrix},
					a \in \mathbb{R}, b \geq 0 \right\}.
				\end{align*}
				The Hessian has range
				\begin{align*} S & = \left\{A \in M_{2,2}(\mathbb{R}) : A =
						\begin{bmatrix}
							8 & 0 \\
							0 & a \\
						\end{bmatrix},
					a \in \mathbb{R} \right\}.
				\end{align*}

			\item Let $\ds{f(x_1,x_2) = x_1^2 + x_2 + 6}$. The domain of $\ds{f}$ is the set  $\ds{D = \mathbb{R}^2}$. The range of $\ds{f}$ is the set $\ds{S = \mathbb{R}}$. Considering the Gradient $\ds{(\nabla)}$ and Hessian $\ds{(H)}$ of $\ds{f(\mathbf{x})}$, 
				\begin{align*}
					\nabla f(\mathbf{x}) & = 
					\begin{bmatrix}
						2x_1 \\
						1 \\
					\end{bmatrix}, \\
					Hf(\mathbf{x}) & = 
					\begin{bmatrix}
						2 & 0 \\
						0 & 0 \\
					\end{bmatrix}.
				\end{align*}
				Both the Gradient and Hessian of $\ds{f(\mathbf{x})}$ have domain $\ds{D = \mathbb{R}^2}$. The Gradient has range 
				\begin{align*} S & = \left\{A \in M_{2,1}(\mathbb{R}) : A =
						\begin{bmatrix}
							a \\
							1 \\
						\end{bmatrix},
					a \in \mathbb{R}\right\}.
				\end{align*}
				The Hessian has range
				\begin{align*} S & = 
					\begin{bmatrix}
						2 & 0 \\
						0 & 0 \\
					\end{bmatrix}.
				\end{align*}

			\item Let $\ds{f(x_1,x_2) = x_1x_2 + x_1^3 -3}$. The domain of $\ds{f}$ is the set  $\ds{D = \mathbb{R}^2}$. The range of $\ds{f}$ is the set $\ds{S = \mathbb{R}}$. Considering the Gradient $\ds{(\nabla)}$ and Hessian $\ds{(H)}$ of $\ds{f(\mathbf{x})}$, 
				\begin{align*}
					\nabla f(\mathbf{x}) & = 
					\begin{bmatrix}
						x_2 + 3x_1^2 \\
						x_1 \\
					\end{bmatrix}, \\
					Hf(\mathbf{x}) & = 
					\begin{bmatrix}
						6x_1 & 1 \\
						1 & 0 \\
					\end{bmatrix}.
				\end{align*}
				Both the Gradient and Hessian of $\ds{f(\mathbf{x})}$ have domain $\ds{D = \mathbb{R}^2}$. The Gradient has range 
				\begin{align*} S & = \left\{A \in M_{2,1}(\mathbb{R}) : A =
						\begin{bmatrix}
							a \\
							b \\
						\end{bmatrix},
					a,b \in \mathbb{R} \right\}.
				\end{align*}
				The Hessian has range
				\begin{align*} S & = \left\{A \in M_{2,2}(\mathbb{R}) : A =
						\begin{bmatrix}
							a & 1 \\
							1 & 0 \\
						\end{bmatrix},
					a \in \mathbb{R} \right\}.
				\end{align*}
		\end{enumerate}

		\item Consider $\ds{f:\mathbb{R}^3 \rightarrow \mathbb{R}^3}$, where 
			\begin{align*}
				f(\mathbf{x}) & = 
				\begin{bmatrix}
					x_1^2 + x_2^3 - x_3^4 \\
					x_1x_2x_3 \\
					2x_1x_2 - 3x_2x_3 + x_1x_3 \\
				\end{bmatrix}.
			\end{align*}
			The Jacobian $\ds{(J)}$ of $\ds{f(\mathbf{x})}$ is 
			\begin{align*}
				Jf(\mathbf{x}) & = 
				\begin{bmatrix}
					2x_1 & 3x_2^2 & -4x_3^3 \\
					x_2x_3 & x_1x_3 & x_1x_2 \\
					2x_1 + x_3 & 2x_1 - 3x_3 & -3x_2 + x_1 \\
				\end{bmatrix}.
			\end{align*}

		\item Consider $\ds{f:\mathbb{R}^2 \rightarrow \mathbb{R}}$, where $\ds{f(\mathbf{x}) = x_1^3x_2^4+ \frac{x_2}{x_1}}$. The Gradient $\ds{(\nabla)}$ and Hessian $\ds{(H)}$ of $\ds{f(\mathbf{x})}$ are
			\begin{align*}
				\nabla f(\mathbf{x}) & = 
				\begin{bmatrix}
					3x_1^2x_2^4 - \frac{x_2}{x_1^2} \\
					\\
					4x_2^3x_1^3 + \frac{1}{x_1} \\
				\end{bmatrix}, \\
				\\
				Hf(\mathbf{x}) & = 
				\begin{bmatrix}
					6x_1x_2^4 - \frac{2x_2}{x_1^3} & 12x_1^2x_2^3 - \frac{1}{x_1^2} \\
					\\
					12x_1^2x_2^3 - \frac{1}{x_1^2} & 12x_2^2x_1^3 \\
				\end{bmatrix}.
			\end{align*}
			Both the Gradient and Hessian of $\ds{f(\mathbf{x})}$ have domain $\ds{D = \{\mathbf{x} \in \mathbb{R}^2 : \mathbf{x} = (x_1,x_2), x_1 \neq 0\}}$. The Gradient has range 
				\begin{align*} S & = \left\{A \in M_{2,1}(\mathbb{R}) : A =
						\begin{bmatrix}
							a \\
							b \\
						\end{bmatrix},
					a,b \in \mathbb{R} \right\}.
				\end{align*}
				The Hessian has range
				\begin{align*} S & = \left\{A \in M_{2,2}(\mathbb{R}) : A =
						\begin{bmatrix}
							a & b \\
							c & d \\
						\end{bmatrix},
					a,b,c,d \in \mathbb{R} \right\}.
				\end{align*}

		\item \begin{enumerate}[label=\alph*)]
			\item Let $\ds{\mathbf{x} = (H,L,W)}$. The box then has a surface area of $\ds{A(\mathbf{x}) = 2HL + 2LW + 2HW}$, and a volume of $\ds{V(\mathbf{x}) = HLW}$. Thus, we can rewrite the formulas given as
				\begin{align*}
					Q(\mathbf{x}) & = KHLW(T-T_a), \\
					h_c(\mathbf{x}) & = k_c(2HL + 2LW + 2HW)(T-T_a), \\
					h_r(\mathbf{x}) & = k_r(2HL + 2LW + 2HW)(T-T_a^4). 
				\end{align*}
				Furthermore, total heat loss, $\ds{T(\mathbf{x})}$, is found by multiplying rate of heat loss per unit of time, by the amount of time elapsed. Thus, we have the following equation for total heat loss, over $\ds{S}$ units of time,
				\begin{align*}
					T(\mathbf{x}) & = S(h_c + h_r) \\
					\therefore T(\mathbf{x}) & = S(2HL + 2LW + 2HW) \left[k_c(T-T_a) + k_r(T-T_a^4)\right].
				\end{align*}
				Thus, to formulate the minimisation problem, we must minimise $\ds{T(\mathbf{x})}$, subject to the conditions 
				\begin{align*}
					Q(\mathbf{x}) & \geq Q^{\prime}, \\
					0 \leq H & \leq H^{\prime}, \\
					0 \leq L & \leq L^{\prime}, \\
					0 \leq W & \leq W^{\prime}.
				\end{align*}

			\item Again, let $\ds{\mathbf{x} = (H,L,W)}$. Considering an insulation thickness of $\ds{t}$, on the outside of the box, the box now has the dimensions $\ds{\mathbf{x}+2t = (H+2t,L+2t,W+2t)}$. Calculating the volume of the insulation, $\ds{V_i(\mathbf{x})}$, requires us to calculate the new volume of the box, with the insulation on the outside, and then subtract the original volume of the box, yielding
				\begin{align*}
					V_i(\mathbf{x}) & = V(\mathbf{x} + 2t) - V(\mathbf{x}) \\
									& = (H+2t)(L+2t)(W+2t) - HLW \\
					\therefore V_i(\mathbf{x}) & = 2t(HL + LW + HW) + 4t^2(H + L + W) + 8t^3.
				\end{align*}
				As cost is proportional to the volume of the insulation, we have $\ds{C_i(\mathbf{x}) = \alpha V_i(\mathbf{x})}$, for some constant $\ds{\alpha}$. Thus, for the cost of the insulation, we have 
				\begin{align*}
					C_i(\mathbf{x}) & = \alpha \left[2t(HL + LW + HW) + 4t^2(H + L + W) + 8t^3\right].
				\end{align*}
				As we may consider the insulation to have negligible effect when computing the surface area and volume for heat loss and heat storage, we may use the equation for $\ds{Q(\mathbf{X})}$ in part $\ds{a)}$, for the conditions. Thus, in formulating the minimisation problem, we must minimise $\ds{C_i(\mathbf{x})}$, subject to the conditions
				\begin{align*}
					Q(\mathbf{x}) & \geq Q^{\prime}, \\
					0 \leq H & \leq H^{\prime}, \\
					0 \leq L & \leq L^{\prime}, \\
					0 \leq W & \leq W^{\prime}.
				\end{align*}
				


		\end{enumerate}

	\end{enumerate}
\end{document}
