% Begin the document and set up the style of the document
\documentclass[a4paper,11pt]{article}

% Install the required packages for the document 
\usepackage{enumitem}
\usepackage{amsmath}
\usepackage{amssymb}
\usepackage{verbatim}
\usepackage{mathtools}
\usepackage{tikz}
\usepackage{nicefrac}
\usepackage{bm}
\usepackage{xlop}

\newcommand{\norm}[1]{\left\lVert#1\right\rVert}


% Page and style settings
%\parskip=8pt
\parindent=0pt
% Right margin
\textwidth=6.25in
% Left margin
\oddsidemargin=0pt
\evensidemargin=0pt
% Bottom margin
\textheight=10in
% Top margin
\topmargin=-0.75in
\baselineskip=11pt
% end of page and other style settings

\renewcommand{\familydefault}{\sfdefault}
\usepackage{calrsfs}
\DeclareMathAlphabet{\pazocal}{OMS}{zplm}{m}{n}

\newcommand{\indep}{\mathrel{\text{\scalebox{1.07}{$\perp\mkern-10mu\perp$}}}}
\newcommand{\p}{\mathbb{P}}
\newcommand{\e}{\mathbb{E}}
\newcommand{\ds}{\displaystyle}
\newcommand{\code}{\texttt}
\newcommand{\HRule}{\rule{\linewidth}{0.5mm}} % Defines a new command for the horizontal lines, change thickness here

\newenvironment{nscentre}
 {\parskip=0pt\par\nopagebreak\centering}
 {\par\noindent\ignorespacesafterend}


\usepackage{fullpage}

\usepackage{titlesec} % Used to customize the \section command
\titleformat{\section}{\bf}{}{0em}{}[\titlerule] % Text formatting of sections
\titlespacing*{\section}{0pt}{3pt}{3pt} % Spacing around sections

\begin{document}
\setlength{\abovedisplayskip}{8pt}{%
\setlength{\belowdisplayskip}{8pt}{%


\text{LSA}
\hfill
\text{University of Michigan}

\begin{nscentre}
	\textbf{ASTRO201: Introduction to Astrophysics}\\
	\textbf{Homework 5}\\
\end{nscentre}

\text{Name: Keegan Gyoery}
\hfill
\text{UM-ID: 31799451}

\pagenumbering{arabic}
	\begin{enumerate}[leftmargin=*]
		\item Cygnus X-1 has a mass of 10 solar masses, and its accretion powered luminosity is given to be $\ds{5\times10^{38}\:}$erg/sec.
			\begin{enumerate}[label=\alph*)]
				\item Using the formula for the Schwarzschild radius, we calculate the radius as
					\begin{align*}
						r_s & = \frac{2GM}{c^2} \\
							& = \frac{2\times 6.674\times10^{-8}\times 10\times2\times 10^{33}}{\left(3\times 10^{10} \right)^2} \\
							& = 2966222.222.
					\end{align*}
					Thus, the Schwarzschild radius of Cygnus X-1 is $\ds{2.97\times 10^6\:}$cm.
				\item Using the formula for the Eddington Luminosity, we have
					\begin{align*}
						L_{\text{edd}} & = 1.3\times10^{38}\times\frac{M}{M_{\odot}} \\
									   & = 1.3\times10^{38}\times\frac{10\times M_{\odot}}{M_{\odot}} \\
									   & = 1.3\times10^{39}.
					\end{align*}
					Thus, accretion powered luminosity provides $\ds{38\%}$ of the Eddington Luminosity.
				\item Using the formula for rest-mass energy conversion in a blackhole, with $\ds{10\%}$ efficiency, we have
					\begin{align*}
						\Delta E & = 0.1\times mc^2 \\
						\therefore m & = \frac{10\times E}{c^2} \\
									 & = \frac{10\times 5\times 10^{38}}{\left(3\times 10^{10} \right)^2} \\
									 & = 5.555555556\times 10^{18}.
					\end{align*}
					Thus, the mass accretion rate onto Cygnus X-1 is $\ds{5.56 \times 10^{18}\:}$g/sec, which is $\ds{8.6\times 10^{-8}\:\text{M}_{\odot}}$/year.
			\end{enumerate}
		\item Assume a typical neutron star, with a radius of $\ds{10\:}$km and a mass of 1.4 solar masses.
			\begin{enumerate}[label=\alph*)]
				\item Using the formula for gravitational binding energy for a star with constant mass density, we have
					\begin{align*}
						U & = \frac{3GM^2}{5R} \\
						  & = \frac{3\times 6.674\times 10^{-8}\times \left(1.4\times 2\times 10^{33} \right)^2}{5\times 10^6} \\
						  & = 3.1394496 \times 10^{53}.
					\end{align*}
					Thus, the gravitational binding energy of the neutron star is $\ds{3.14 \times 10^{53}\:}$erg. Using the energy, rest-mass formula, we have
					\begin{align*}
						E & = mc^2 \\
						  & = 1.4\times 2\times 10^{33} \times \left(3\times 10^{10}\right)^2 \\
						  & = 2.52\times 10^{54}.
					\end{align*}
					Thus, the rest-mass energy of the neutron star is $\ds{2.52\times 10^{54}\:}$erg. Clearly, the neutron star's gravitational binding energy is $\ds{20.7\%}$ of the rest-mass energy of the neutron star.
				\item If the energy of a neutron is $\ds{1.5\times 10^{-5}\:}$erg, and the energy released is $\ds{3.14\times 10^{53}\:}$, then the number of neutrinos released is $\ds{2.09\times 10^{58}}$.
				\item Using the equation for dispersion of radiation over a distance, we have 
					\begin{align*}
						\text{\# neutrinos/m}^2 & = \frac{\text{\# neutrinos}}{4\pi R^2} \\
												& = \frac{2.09\times 10^{58}}{4\pi \left(1.543\times 10^24 \right)^2} \\
												& = 698561376.8.
					\end{align*}
					Thus, the number of neutrinos per $\ds{\text{m}^2}$ at Earth is $\ds{6.99 \times 10^8}$. A $\ds{10\:}$m by $\ds{10\:}$m box has a surface area of $\ds{100\:\text{m}^2}$, and so $\ds{6.99 \times 10^{10}}$ neutrinos pass through the face of the box.
			\end{enumerate}
		\item Assuming a star has the same size, that is radius, as the Sun, and the same magnetic field of the Sun, 5 Gauss, using the conservation of magnetic fields, when the star collapses to a neutron star of radius $\ds{10\:}$km, its magnetic field becomes
			\begin{align*}
				B_{\odot}R_{\odot}^2 & = BR^2 \\
				\therefore B & = B_{\odot}\left(\frac{R_{\odot}}{R}\right)^2 \\
							 & = 5\times \left(\frac{6.955\times 10^{10}}{10^6}\right)^2 \\
							 & = 2.41860125\times 10^{10} 
			\end{align*}
			Thus, the magnetic field of the neutron star after the collapse is $\ds{2.42\times 10^{10}\:}$Gauss. The LHC produces magnetic fields of up to 8.36 Tesla, which is equivalent to $\ds{8.36\times 10^4\:}$Gauss, which is approximately 300,000 times smaller.
		\item In order for the rotating star to remain intact, the force of gravity of the star must exceed the centrifugal force exerted by rotation. Thus, knowing $\ds{w = \frac{2\pi}{P}}$, we have the following inequality,
			\begin{align*}
				\frac{GMm}{R^2} & > mRw^2 \\
				\frac{GM}{R^3} & > w^2 \\
				\frac{GM}{R^3} & >  \left(\frac{2\pi}{P}\right)^2 \\
				\frac{GM}{R^3} & > \frac{4\pi^2}{P^2} \\
				\frac{3M}{4\piR^3} & > \frac{3\pi}{GP^2} \\
				\rho & > \frac{3\pi}{P^2}.
			\end{align*}
			Clearly, the rotation period of a star translates into a density. If the star has a rotation period of 1 millisecond, or $\ds{10^{-3}\:}$seconds, the minimum density is
			\begin{align*}
				\rho & > \frac{3\pi}{GP^2} \\
				\therefore \rho_{\text{min}} & = \frac{3\pi}{6.674\times 10^{-8} \times \left(10^{-3}\right)^2} \\
											 & = 1.412163314\times 10^{14}
			\end{align*}
			Thus, the minimum density of the star is $\ds{1.41\times 10^{14}\:\text{g/cm}^3}$.
	\end{enumerate}
\end{document}
