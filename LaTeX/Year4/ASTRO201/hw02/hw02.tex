% Begin the document and set up the style of the document
\documentclass[a4paper,11pt]{article}

% Install the required packages for the document 
\usepackage{enumitem}
\usepackage{amsmath}
\usepackage{amssymb}
\usepackage{verbatim}
\usepackage{mathtools}
\usepackage{tikz}
\usepackage{nicefrac}
\usepackage{bm}
\usepackage{xlop}

\newcommand{\norm}[1]{\left\lVert#1\right\rVert}


% Page and style settings
%\parskip=8pt
\parindent=0pt
% Right margin
\textwidth=6.25in
% Left margin
\oddsidemargin=0pt
\evensidemargin=0pt
% Bottom margin
\textheight=10in
% Top margin
\topmargin=-0.75in
\baselineskip=11pt
% end of page and other style settings

\renewcommand{\familydefault}{\sfdefault}
\usepackage{calrsfs}
\DeclareMathAlphabet{\pazocal}{OMS}{zplm}{m}{n}

\newcommand{\indep}{\mathrel{\text{\scalebox{1.07}{$\perp\mkern-10mu\perp$}}}}
\newcommand{\p}{\mathbb{P}}
\newcommand{\e}{\mathbb{E}}
\newcommand{\ds}{\displaystyle}
\newcommand{\code}{\texttt}
\newcommand{\HRule}{\rule{\linewidth}{0.5mm}} % Defines a new command for the horizontal lines, change thickness here

\newenvironment{nscentre}
 {\parskip=0pt\par\nopagebreak\centering}
 {\par\noindent\ignorespacesafterend}


\usepackage{fullpage}

\usepackage{titlesec} % Used to customize the \section command
\titleformat{\section}{\bf}{}{0em}{}[\titlerule] % Text formatting of sections
\titlespacing*{\section}{0pt}{3pt}{3pt} % Spacing around sections

\begin{document}
\setlength{\abovedisplayskip}{8pt}{%
\setlength{\belowdisplayskip}{8pt}{%


\text{LSA}
\hfill
\text{University of Michigan}

\begin{nscentre}
	\textbf{ASTRO201: Introduction to Astrophysics}\\
	\textbf{Homework 2}\\
\end{nscentre}

\text{Name: Keegan Gyoery}
\hfill
\text{UM-ID: 31799451}

\pagenumbering{arabic}
	\begin{enumerate}[leftmargin=*]
		\item \begin{enumerate}[label=\alph*)]
			\item Using the formula for the energy of level $\ds{n}$,
				\begin{align*}
					E_n & = \frac{-13.6}{n^2} \\
					E_1 & = -13.6 \:\text{eV}, \\
					E_2 & = -3.4 \:\text{eV}, \\
					E_3 & = -1.5 \:\text{eV}, \\
					E_4 & = -0.85 \:\text{eV}, \\
					E_5 & = -0.54 \:\text{eV}.
				\end{align*}
			\item Using the previous part, the energies of the first $\ds{3}$ transitions are
				\begin{align*}
					E_2 - E_3 & = -1.9 \:\text{eV}, \\
					E_2 - E_4 & = -2.55 \\
							  & = -2.6 \:\text{eV}, \\
					E_2 - E_5 & = -2.856 \\
							  & = -2.9 \:\text{eV}.
				\end{align*}
			\item Rearranging the formula relating energy and wavelength, with $\ds{E_m > E_n}$, 
				\begin{align*}
					E_m - E_n & = \frac{hc}{\lambda} \\
					\therefore \lambda & = \frac{hc}{E_m - E_n}. \\
					\lambda & = \frac{hc}{E_3 - E_2} \\
							& = \frac{4.135667696 \times 10^{-15} \times 3 \times 10^{17}}{1.9} \\
					\therefore \lambda & = 653 \: \text{nm}, \\
					\lambda & = \frac{hc}{E_4 - E_2} \\
							& = \frac{4.135667696 \times 10^{-15} \times 3 \times 10^{17}}{2.55} \\
							& = 486.549 \\
					\therefore \lambda & = 490 \: \text{nm}, \\
					\lambda & = \frac{hc}{E_5 - E_2} \\
							& = \frac{4.135667696 \times 10^{-15} \times 3 \times 10^{17}}{2.856} \\
							& = 434.4188756 \\
					\therefore \lambda & = 430 \: \text{nm}.
				\end{align*}
			\item These Balmer series transitions fall under the Visible Light part of the spectrum.
		\end{enumerate}
		\item \begin{enumerate}[label=\alph*)]
			\item From the previous part, the $\ds{n=4}$ energy level has an energy of $\ds{-0.85\:}$eV, and so a hydrogen atom with its electron in the $\ds{n=4}$ level requires $\ds{0.85\:}$eV to ionise the atom.
			\item Using the formula relating temperature and energy of particles,
				\begin{align*}
					\kappa T & = h \nu \\
							 & = E_m - E_n \\
					\therefore T & = \frac{E_m - E_n}{\kappa} \\
					T & = \frac{E_{\infty} - E_4}{\kappa} \\
					  & = \frac{0 - (-0.85)}{8.617333262145\times10^{-5}} \\
					  & = 9863.84 \\
					\therefore T & = 9900\: \text{K}
				\end{align*}
		\end{enumerate}
		\item \begin{enumerate}[label=\alph*)]
			\item Note the conversions $\ds{100\:\text{AU} = 1.496\times 10^{13}\: \text{m}}$, and $\ds{745\:\text{yrs} = 2.349 \times 10^{10}\: \text{s}}$. Using the formula for the sum of masses in an edge-on binary system, 
				\begin{align*}
					\frac{4\pi^2 R^3}{G} & = (m_1+m_2)P^2 \\
					\therefore m_1 + m_2 & = \frac{4\pi^2 R^3}{G\times P^2} \\
										 & = \frac{4\pi^2 \left(1.496\times 10^{13}\right)^3}{6.674\times 10^{-11}\times \left(2.349\times 10^{10}\right)^2} \\
										 & = 3.589238201\times 10^30\:\text{kg} \\
										 & = 1.804544093\:\text{M}_{\odot} \\
					\therefore m_1 + m_2 & = 1.8 \:\text{M}_{\odot}
				\end{align*}
			\item As we have been given the maximum Doppler shift of a $\ds{500\:}$nm spectral line, this occurs when the angle of either star relative to the centre of their orbit is $\ds{0}$. Thus, the formula used is
				\begin{align*}
					\frac{\Delta \lambda}{\lambda} & = \frac{v}{c} \cos({0}) \\
					\therefore v & = \frac{c\Delta \lambda}{\lambda}, \\
					v_1 & = \frac{c\Delta \lambda}{\lambda} \\
						& = \frac{3\times 10^{10} \times 4.05\times 10^{-10}}{500\times 10^{-7}} \\
						& = 243000\:\text{cm/s} \\
						& = 0.5123\:\text{AU/yr} \\
					\therefore v_1 & = 0.51\:\text{AU/yr}, 	
				\end{align*}
				\pagebreak

				\begin{align*}
					v_2 & = \frac{c\Delta \lambda}{\lambda} \\
						& = \frac{3\times 10^{10} \times 2.59\times 10^{-10}}{500\times 10^{-7}} \\
						& = 155400\:\text{cm/s} \\
						& = 0.3276\:\text{AU/yr} \\
					\therefore v_2 & = 0.33\:\text{AU/yr}, 	
				\end{align*}
			\item Combining the formulas relating each star's mass, radius and velocity in an edge-on binary system,
				\begin{align*}
					m_1r_1 & = m_2r_2 \\
					\frac{r_1}{r_2} & = \frac{v_1}{v_2} \\
					\therefore \frac{m_2}{m_1} & = \frac{v_1}{v_2} \\
											   & = \frac{0.5123}{0.3276} \\
											   & = 1.563797314 \\
					\therefore \frac{m_2}{m_1} & = 1.6.
				\end{align*}
				Thus, the ratio of Star 1's mass to Star 2's mass is $\ds{1 : 1.6}$.
			\item Using the two equations found in the previous parts, and solving simultaneously,
				\begin{align*}
					m_1 + m_2 & = 1.804544093 \dots (A) \\
					\frac{m_2}{m_1} & = 1.563797314 \dots (B) \\
					(B) & \Rightarrow \: m_2 = 1.563797314 m_1 \dots (C) \\
					(C) \rightarrow (B) & \Rightarrow \: 2.563797314 m_1 = 1.804544093 \\
					\therefore m_1 & = 0.703855988 \\
								   & = 0.70 \:\text{M}_{\odot} \\
					\therefore m_2 & = 1.100688105 \\
								   & = 1.1 \:\text{M}_{\odot} \\
				\end{align*}
		\end{enumerate}
		\pagebreak
		\item \begin{enumerate}[label=\alph*)]
			\item Converting the apparent magnitudes to luminosities, we have,
				\begin{align*}
					\frac{L_{binary}}{L_0} & = \frac{L_{+2.4}}{L_0} + \frac{L_{+5.2}}{L_0} \\
										   & = 10^{\nicefrac{-2.4}{2.5}} + 10^{\nicefrac{-5.2}{2.5}} \\
										   & = 0.109647819 + 0.008317637 \\
										   & = 0.117965456 \\
					m_{binary} & = -2.5\log{\left(\frac{L_{binary}}{L_0}\right)} \\
							   & = -2.5\log{0.117965456} \\
							   & = 2.320612873 \\
					\therefore m_{binary} & = 2.3 
				\end{align*}
			\item For Star A, using the Appendix, we have an apparent magnitude of $\ds{+5.20}$, and an absolute magnitude of $\ds{+7.4}$. Thus,
				\begin{align*}
					m_V - M_V & = -2.2 \\
					\therefore 5\log_{10}d_{pc} - 5 & = -2.2 \\
					\log_{10}d_{pc} & = 0.56 \\
					d_{pc} & = 10^{0.56} \\
						   & = 3.63 \\
					\therefore d_{pc} & = 3.6\:\text{parsecs}.
				\end{align*}
			For Star B, using the Appendix, we have an apparent magnitude of $\ds{+2.4}$, and an absolute magnitude of $\ds{+4.6}$. Thus,
				\begin{align*}
					m_V - M_V & = -2.2 \\
					\therefore 5\log_{10}d_{pc} - 5 & = -2.2 \\
					\log_{10}d_{pc} & = 0.56 \\
					d_{pc} & = 10^{0.56} \\
						   & = 3.63 \\
					\therefore d_{pc} & = 3.6\:\text{parsecs}.
				\end{align*}
				Thus, both stars are $\ds{3.6\:}$parsecs away, and so likely are in a binary system.

		\end{enumerate}
	\end{enumerate}
\end{document}
