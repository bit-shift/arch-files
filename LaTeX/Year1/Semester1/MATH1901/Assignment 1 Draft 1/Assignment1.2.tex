\documentclass[a4paper]{article}
% you can copy and paste these page settings from here to the next % to make the pdf look a little nicer
\usepackage{geometry}
\usepackage{enumitem}
\usepackage{mathtools}
\usepackage{graphicx}
\usepackage{amsmath}
\usepackage{amscd}
\usepackage{amssymb}
\usepackage{amsfonts}
\usepackage{pgf,tikz}
\usepackage{mathrsfs}
\usepackage{asyalign}
\usepackage[tableposition=top]{caption}
\usepackage{ifthen}
\usepackage[utf8]{inputenc}{}
\usetikzlibrary{arrows}

% page and style settings
\parskip=5pt
\parindent=0pt
\textwidth=6.25in
\oddsidemargin=0pt
\evensidemargin=0pt
\textheight=10in
\topmargin=-0.75in
\baselineskip=11pt
% end of page and other style settings

\renewcommand{\familydefault}{\sfdefault}

\begin{document}

% the following four lines force that lines break after 80 characters in the R-output


%%%%%%%%%%%%%%%%%%%%%%%%%%%%%%%% HEADER FOR PDF %%%%%%%%%%%%%%%%%%%%%%%%%%%%%%%%%
%%%%%%%%%%%%%%%%%%%%%%%%%%%% (Ignore this section) %%%%%%%%%%%%%%%%%%%%%%%%%%%%%%

\begin{center}
{\large \textbf{MATH 1901 - Differential Calculus} }\\
\end{center}
\vspace{-1mm}
\begin{tabular*}{1.0\linewidth}{@{\extracolsep{\fill}}lr@{}}
  \hline\noalign{\smallskip}
Semester 1, \the\year & Name = \texttt{Keegan Gyoery} \\ 
Tutor = \texttt{Daniel Daners} & SID = \texttt{470413467} \\
\hline
\end{tabular*}
\begin{center}
 \large \textbf{Assignment 1}\\
\end{center}
%%%%%%%%%%%%%%%%%%%%%%%%%%%%%%%% END HEADER %%%%%%%%%%%%%%%%%%%%%%%%%%%%%%%%%%%%%
%%%%%%%%%%%%%%%%%%%%%%%%%%%%%%%%%%%%%%%%%%%%%%%%%%%%%%%%%%%%%%%%%%%%%%%%%%%%%%%%%
%%%%%%%%%%%%%%%%%%%%%%%%%% QUESTIONS START HERE %%%%%%%%%%%%%%%%%%%%%%%%%%%%%%%%%
%%%%%%%%%%%%%%%%%%%%%%%%%%%%%%%%%%%%%%%%%%%%%%%%%%%%%%%%%%%%%%%%%%%%%%%%%%%%%%%%%
% Bold heading Question 1
\textbf{Question 1}

% Begin List 
\begin{enumerate}[label=(\alph*)]

% First Question
\item Consider the equation\\

% Equation we are required to prove
$x^{n+1}-(n+1)x+n = (x-1)[1+x+x^2+...+x^n-(n+1)$\\
We are required to prove this result for the conditions:\\

% Conditions for the above equation to hold
$x\geq1, x\in\mathbb{R}, n\in\mathbb{N}$\\


% Proving the result RHS
$RHS = (x-1)[1+x+x^2+...+x^n-(n+1)]$\\

% Setting up the equation as a GP
It can be seen that $1+x+x^2+...+x^n$ is a GP, with first term = $1$, ratio = $x$ , and $(n+1)$ terms\\

% Solving the equation
$\therefore 1+x+x^2+...+x^n = \frac{(x^{n+1}-1)}{x-1}$\\

% Begin align and solving the equation














\end{enumerate}

\end{document}
