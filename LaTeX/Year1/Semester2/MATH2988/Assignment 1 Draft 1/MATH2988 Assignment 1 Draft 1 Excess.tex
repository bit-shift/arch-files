In order to complete the proof, we first will assume that there exists a $\displaystyle{c \in \mathbb{Z}^{+}}$, such that $\displaystyle{c}$ is a solution to the above polynomial congruence relation. Furthermore, we will construct the following definition.
		\begin{align*}
		x^n & \equiv x+1 \pmod{p}\\
		\therefore x^n - x - 1 & \equiv 0 \pmod{p}\\
		f(x) & \coloneqq x^n - x -1 \\
		\therefore f(x) & \equiv 0 \pmod{p}\\
		\end{align*}
		Furthermore, we are able to eliminate the cases for $\displaystyle{n = 1}$ and $\displaystyle{n = p-1}$. We shall first examine the case where $\displaystyle{n=1}$.
		\begin{align*}
		x^{n} & \equiv x + 1 \pmod{p}\\
		x^{1} & \equiv x + 1 \pmod{p}\\
		\therefore x & \equiv x + 1 \pmod{p}\\
		\end{align*}
		This is obviously not possible, and thus the result does not hold for $\displaystyle{n=1}$. Now considering the case where $\displaystyle{n = p-1}$.
		\begin{align*}
		x^{n} & \equiv x + 1 \pmod{p}\\
		x^{p-1} & \equiv x + 1 \pmod{p}\\
		x^{p-1} & \equiv 1 \pmod{p}\\
		\therefore 1 & \equiv x + 1 \pmod{p}\\
		\end{align*}
		Once again, this is obviously not possible, and thus the result does not hold for $\displaystyle{n = p-1}$.

		\bigbreak

		As a result of this assumption of the existence of $\displaystyle{c}$, we have the following results.
		\begin{align*}
		f(c) & \equiv 0 \pmod{p}\\
		\therefore f(x) - f(c) & \equiv 0 \pmod{p}\\
		\therefore x^n - x - 1 - c^n + c + 1 & \equiv 0 \pmod{p}\\
		\therefore x^n - c^n - x + c & \equiv 0 \pmod{p}\\
		\therefore (x-c)\big[x^{n-1} + x^{n-2}y + \dots + xy^{n-2} + y^{n-1} \big] - (x-c) & \equiv 0 \pmod{p}\\
		\therefore (x-c)\big[x^{n-1} + x^{n-2}y + \dots + xy^{n-2} + y^{n-1} - 1 \big] & \equiv 0 \pmod{p}\\
		\end{align*}
		As a result, we have that $\displaystyle{p \mid (x-c)}$. Thus we get the following consequences.
		\begin{align*}
		x-c & \equiv 0 \pmod{p}\\
		\therefore x & \equiv c \pmod{p}\\
		\end{align*}
		Thus there exists an integer solution $\displaystyle{x}$ for all $\displaystyle{n \in \{2,3,\dots,p-2 \}}$.

		\bigbreak

		ADD IN THE CONDITION ABOUT P GREATER THAN 3.



In order to complete this proof, we shall use Fermat's factorisation method, which allows us to factorise any integer as the difference of two squares. That is we have the result $\displaystyle{n^2 + 1 = (p+q)(p-q)}$, where $\displaystyle{p,q \in \mathbb{Z}^+}$. Thus we can set the following results.
		\begin{align*}
		a & = p+q \\
		b & = p-q \\
		\end{align*}
		Now consider $\displaystyle{a}$ and $\displaystyle{b}$, and their nature as either odd or even. We have 3 cases to consider.
		\begin{itemize}
		\item $\displaystyle{a}$ and $\displaystyle{b}$ are both even \dots\dots\dots\dots (1)
		\item $\displaystyle{a}$ and $\displaystyle{b}$ are both odd \dots\dots\dots\dots (2)
		\item One odd and one even factor \dots\dots (3)
		\end{itemize}




In approaching this problem, the main issue to overcome is the fact that the algorithm will require repition in order to solve for $\displaystyle{m}$. As a result, a naive approach cannot be taken, as the algorithm will not be polynomial time.