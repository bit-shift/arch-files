% Begin the document and set up the style of the document
\documentclass[a4paper]{article}

% Install the required packages for the document 
\usepackage{envmath}
\usepackage{esvect}
\usepackage{graphicx}
\usepackage{gensymb}
\usepackage{tikz}
\usepackage[mathcal]{euscript}
\usepackage{geometry}
\usepackage{enumitem}
\usepackage{mathtools}
\usepackage{graphicx}
\usepackage{amsmath}
\usepackage{amscd}
\usepackage{amssymb}
\usepackage{amsfonts}
\usepackage{harpoon}
\usepackage{pgf}
\usepackage{tikz}
\usepackage{mathrsfs}
\usepackage{asyalign}
\usepackage{physics}
\usepackage{enumitem}
\usepackage{xhfill}
\usepackage{accents}
\usepackage{cite}
\usepackage{url}
\usepackage[tableposition=top]{caption}
\usepackage{ifthen}
\usepackage[utf8]{inputenc}
\usepackage{tikz-3dplot}
\usepackage{eulervm}
\usetikzlibrary{patterns}
\usetikzlibrary{arrows}

% Page and style settings
\parskip=8pt
\parindent=0pt
% Right margin
\textwidth=6.25in
% Left margin
\oddsidemargin=0pt
\evensidemargin=0pt
% Bottom margin
\textheight=10in
% Top margin
\topmargin=-0.75in
\baselineskip=11pt
% end of page and other style settings

\renewcommand{\familydefault}{\sfdefault}


% Begin the text of the document
\begin{document}

% Begin the Title Page
\begin{titlepage}

\newcommand{\HRule}{\rule{\linewidth}{0.5mm}} % Defines a new command for the horizontal lines, change thickness here

\center % Center everything on the page
 
\textsc{\LARGE University of Sydney}\\[1.5cm] % Name of your university/college
\textsc{\Large MATH 1903}\\[0.5cm] % Major heading such as course name
\textsc{\large Integral Calculus and Modelling Advanced}\\[0.5cm] % Minor heading such as course title

\HRule \\[0.4cm]
{ \huge \bfseries Assignment 1}\\[0.4cm] % Title of your document
\HRule \\[1.5cm]

\begin{minipage}{0.4\textwidth}
\begin{flushleft} \large
\emph{Author:}
Keegan Gyoery % Your name
\\
\emph{SID:}
470413467
\end{flushleft}
\end{minipage}
~
\begin{minipage}{0.4\textwidth}
\begin{flushright} \large
\emph{Tutor:} 
Dr Daniel Daners % Tutor's Name
\\
\emph{Tutorial:}
Carslaw 361
Tuesday 1pm
\end{flushright}
\end{minipage}\\[4cm]

{\large \today}\\[3cm] % Date, change the \today to a set date if you want to be precise

\vfill % Fill the rest of the page with whitespace

\end{titlepage}

\pagenumbering{arabic}

\begin{enumerate}[label=\textbf{\arabic*.}]
	\item Let $\displaystyle{f \in \mathbb{R}^{\Delta}}$ and define $\displaystyle{g,h \in \mathbb{R}^{\Delta}}$ by the rules
	\begin{align*}
	g(x) & = \frac{f(x) + f(-x)}{2}\\
	h(x) & = \frac{f(x) - f(-x)}{2}\\
	\end{align*}
	for all $\displaystyle{x \in {\Delta}}$. 

	\bigbreak

	We are now asked to prove which function out of $g$ and $h$ is even and which is odd. The definition of an even function is $\displaystyle{f(-x) = f(x)}$, and the definition of an odd function $\displaystyle{f(-x) = -f(x)}$. Both of these defintions will be used to determine the nature of the functions $g$ and $h$.

	\bigbreak

	Firstly, examining the function $g(x)$, we get the following results.

	\begin{align*}
	g(x) & = \frac{f(x) + f(-x)}{2}\\
	\therefore g(-x) & = \frac{f(-x) + f(-(-x))}{2} \hspace{5mm} \text{substituting $\displaystyle{-x}$}\\
	& = \frac{f(-x) + f(x)}{2}\\
	& = \frac{f(x) + f(-x)}{2}\\
	\therefore g(-x) & = g(x)\\
	\end{align*}

	Therefore, the function $g(x)$ is an even function as it satisfies the above defintion of an even function.

	\bigbreak

	Secondly, examining the function $h(x)$, we get the following results.

	\begin{align*}
	h(x) & = \frac{f(x) - f(-x)}{2}\\
	\therefore h(-x) & = \frac{f(-x) - f(-(-x))}{2} \hspace{5mm} \text{substituting $\displaystyle{-x}$}\\
	& = \frac{f(-x) - f(x)}{2}\\
	& = \frac{-f(x) + f(-x)}{2}\\
	& = \frac{-\Big[f(x) - f(-x)\Big]}{2}\\
	\therefore h(-x) & = -h(x)\\
	\end{align*}

	Therefore, the function $h(x)$ is an odd function as it satisfies the above defintion of an odd function.

	\pagebreak

	\item Let $\displaystyle{f \in \mathbb{R}^{\Delta}}$. We are required to prove that

	\begin{align*}
	f = f_{even} + f_{odd}\\
	\end{align*}

	for some unique functions, $\displaystyle{f_{even}},{f_{odd} \in \mathbb{R}^{\Delta}}$ such that $\displaystyle{f_{even}}$ is even and $\displaystyle{f_{odd}}$ is odd.

	\bigbreak

	Using the functions from the previous question $g(x)$ and $h(x)$, we can define two functions, $\displaystyle{f_{even}}$ and $\displaystyle{f_{odd}}$, that are both unique and will satisfy the above condition. If we use the following definitions, the required result will become obvious.

	\begin{align*}
	f_{even} & \coloneqq \frac{f(x) + f(-x)}{2}\\
	f_{odd} & \coloneqq \frac{f(x) - f(-x)}{2}\\
	\therefore f_{even} + f_{odd} & = \frac{f(x) + f(-x)}{2} + \frac{f(x) - f(-x)}{2}\\
	& = \frac{f(x)}{2} + \frac{f(-x)}{2} + \frac{f(x)}{2} - \frac{f(-x)}{2}\\
	& = \frac{f(x)}{2} + \frac{f(x)}{2}\\
	& = 2\frac{f(x)}{2}\\
	& = f(x)\\
	\therefore f_{even} + f_{odd} & = f(x)\\
	\therefore f & = f_{even} + f_{odd}\\
	\end{align*}

	\item In order to find simplified expressions for $\displaystyle{f_{even}(x)}$ and $\displaystyle{f_{odd}(x)}$, in each of the following cases, we will use both the results from parts $1$ and $2$, in order to construct unique expressions for the $\displaystyle{f(x)}$ given in the question.

	\begin{enumerate}
		
		\item For the first case we have $\displaystyle{\Delta = \mathbb{R}}$, and $\displaystyle{f(x)=e^x}$ for all $\displaystyle{x \in \Delta}$.

		\begin{align*}
		f(x) & = e^x\\
		f_{even}(x) & = \frac{f(x) + f(-x)}{2}\\
		\therefore f_{even}(x) & = \frac{e^x + e^{-x}}{2}\\
		f_{odd}(x) & = \frac{f(x) - f(-x)}{2}\\
		\therefore f_{odd}(x) & = \frac{e^x - e^{-x}}{2}\\
		\end{align*}

		\pagebreak

		\item For the second case we have $\displaystyle{\Delta = \mathbb{R}\setminus \{\pm 1\}}$, and $\displaystyle{f(x)=\frac{1}{1-x}}$ for all $\displaystyle{x \in \Delta}$.

		\begin{align*}
		f(x) & = \frac{1}{1-x}\\
		f_{even}(x) & = \frac{f(x) + f(-x)}{2}\\
		& = \frac{\frac{1}{1-x} + \frac{1}{1+x}}{2}\\
		& = \frac{\frac{1+x+1-x}{(1-x)(1+x)}}{2}\\
		& = \frac{\frac{2}{1-x^2}}{2}\\
		& = \frac{1}{1-x^2}\\
		\therefore f_{even}(x) & = \frac{1}{1-x^2}\\
		f_{odd}(x) & = \frac{f(x) - f(-x)}{2}\\
		& = \frac{\frac{1}{1-x} - \frac{1}{1+x}}{2}\\
		& = \frac{\frac{1+x-1+x}{(1-x)(1+x)}}{2}\\
		& = \frac{\frac{2x}{1-x^2}}{2}\\
		& = \frac{x}{1-x^2}\\
		\therefore f_{odd}(x) & = \frac{x}{1-x^2}\\
		\end{align*}

		\pagebreak

		\item For the third case we have $\displaystyle{\Delta = \mathbb{R} \setminus \mathbb{Z}}$, and $\displaystyle{f(x) = \lfloor{x}\rfloor}$ for all $\displaystyle{x \in \Delta}$. Due to the nature of the floor function and the defintion of the set in which $\displaystyle{\Delta}$ lies, we have the following facts for $\displaystyle{x}$ and $\displaystyle{\lfloor{x}\rfloor}$. Firstly, due to the definition of the set in which $\displaystyle{x}$ lies, that is $\displaystyle{x \in \Delta}$, where $\displaystyle{\Delta \in \mathbb{R} \setminus \mathbb{Z}}$, we have the following result.

		\begin{align*}
		n < x & < n+1 \hspace{5mm} \text{for some $\displaystyle{n \in \mathbb{Z}}$}\\
		\therefore \lfloor{x}\rfloor & = n\\
		\therefore -n > -x & > -(n+1)\\
		\therefore -(n+1) < -x & < -n\\
		\therefore \lfloor{-x}\rfloor & = -(n+1)\\
		\therefore \lfloor{x}\rfloor + \lfloor{-x}\rfloor & = n - (n+1)\\
		& = n - n - 1\\
		& = -1\\
		\therefore \lfloor{x}\rfloor + \lfloor{-x}\rfloor & = -1 \dots\dots(*)\\
		\end{align*}

		From this result we can find the necessary functions $\displaystyle{f_{even}(x)}$ and $\displaystyle{f_{odd}(x)}$, which are as follows.

		\begin{align*}
		f(x) & = \lfloor{x}\rfloor\\
		f_{even}(x) & = \frac{f(x) + f(-x)}{2}\\
		& = \frac{\lfloor{x}\rfloor + \lfloor{-x}\rfloor}{2}\\
		& = -\frac{1}{2} \hspace{5mm} \text{using $\displaystyle{(*)}$}\\
		\therefore f_{even}(x) & = -\frac{1}{2}\\
		f_{odd}(x) & = \frac{f(x) - f(-x)}{2}\\
		& = \frac{\lfloor{x}\rfloor - \lfloor{-x}\rfloor}{2}\\
		& = \frac{\lfloor{x}\rfloor - \Big[ -1 -\lfloor{x}\rfloor \Big]}{2} \hspace{5mm} \text{using $\displaystyle{(*)}$}\\
		& = \frac{\lfloor{x}\rfloor  + 1 + \lfloor{x}\rfloor}{2}\\
		& = \frac{2\lfloor{x}\rfloor  + 1}{2}\\
		& = \lfloor{x}\rfloor + \frac{1}{2}\\
		\therefore f_{odd}(x) & = \lfloor{x}\rfloor + \frac{1}{2}\\
		\end{align*}

	\end{enumerate}

	\pagebreak

	\item For the following proofs, we define $\displaystyle{f \in \mathbb{R}^{\mathbb{R}}}$, and $f$ continuous such that all definite integrals $\displaystyle{\int_{a}^{b}f(x)dx}$ exist and are defined for all  $\displaystyle{a,b \in \mathbb{R}}$.

	\bigbreak

	Firstly, we shall determine the substitution $\displaystyle{t=-u}$, for use in the proofs that follow in $\displaystyle{4a}$, and $\displaystyle{4b}$.

		\begin{align*}
		t & = -u\\
		t = x & \implies u = -x\\
		t = -x & \implies u = x\\
		t = a & \implies u = -a\\
		t = -a & \implies u = a\\
		t = 0 & \implies u = 0\\
		\frac{dt}{du} & = -1\\
		\therefore dt & = -du\\
		\end{align*}

	\begin{enumerate}

		\item Using the properties of integrals, we will construct the following two proofs to show that, for all $\displaystyle{a \in \mathbb{R}}$,

		\begin{align*}
		\int_{-a}^{a}f(x)dx & = 
		\begin{cases}
		\displaystyle{2\int_{0}^{a}f(x)dx} \hspace{5mm} \text{if $f$ is even}\\
		\displaystyle{0} \hfill \text{if $f$ is odd}\\
		\end{cases}
		\end{align*}

		\bigbreak

		For the first case, $f$ is even, and so we construct the following proof.

		\begin{align*}
		\int_{-a}^{a}f(x)dx & = \int_{-a}^{0}f(x)dx + \int_{0}^{a}f(x)dx\\
		& = \int_{a}^{0}f(-t)(-dt) + \int_{0}^{a}f(x)dx \hspace{5mm} \text{substituting $x = -t$}\\
		& = -\int_{a}^{0}f(t)dt + \int_{0}^{a}f(x)dx \hspace{5mm} \text{as $f(x)$ is even}\\
		& = \int_{0}^{a}f(t)dt + \int_{0}^{a}f(x)dx\\
		& = \int_{0}^{a}f(x)dx + \int_{0}^{a}f(x)dx \hspace{5mm} \text{dummy variable switch}\\
		\therefore \int_{-a}^{a}f(x)dx & = 2\int_{0}^{a}f(x)dx \hspace{5mm} \text{if $f$ is even}\\
		\end{align*}

		And thus the first result is proven. 

		\pagebreak

		Now for the second case, $f$ is odd, and so we construct the following proof.

		\begin{align*}
		\int_{-a}^{a}f(x)dx & = \int_{-a}^{0}f(x)dx + \int_{0}^{a}f(x)dx\\
		& = \int_{a}^{0}f(-t)(-dt) + \int_{0}^{a}f(x)dx \hspace{5mm} \text{substituting $x = -t$}\\
		& = -\int_{a}^{0}\big[-f(t)\big]dt + \int_{0}^{a}f(x)dx \hspace{5mm} \text{as $f(x)$ is odd}\\
		& = \int_{a}^{0}f(t)dt + \int_{0}^{a}f(x)dx\\
		& = -\int_{0}^{a}f(t)dt + \int_{0}^{a}f(x)dx\\
		& = -\int_{0}^{a}f(x)dx + \int_{0}^{a}f(x)dx \hspace{5mm} \text{dummy variable switch}\\
		\therefore \int_{-a}^{a}f(x)dx & = 0 \hspace{5mm} \text{if $f$ is odd}\\
		\end{align*}

		And thus the second and final result is proven.

		\pagebreak

		\item For this question, we fix $\displaystyle{a \in \mathbb{R}}$ and define the area function $\displaystyle{A \in \mathbb{R}^{\mathbb{R}}}$ by the rule 
		\begin{align*}
		A(x) & = \int_{a}^{x}f(t)dt\\
		\end{align*}
		for all $\displaystyle{x \in \mathbb{R}}$.

		\bigbreak

		\begin{enumerate}

		\item We are now required to prove that $\displaystyle{A}$ is even if and only if $\displaystyle{f}$ is odd. In order to complete this proof, we must prove the result from both sides of the implication. For the first part of the proof, we are required to prove $\displaystyle{f}$ is odd if $\displaystyle{A}$ is even.

		\begin{align*}
		A(x) & = A(-x)\\
		\therefore A(x) - A(-x) & = 0\\
		\therefore LHS & = \int_{a}^{x}f(t)dt - \int_{a}^{-x}f(t)dt\\
		& = \int_{a}^{0}f(t)dt + \int_{0}^{x}f(t)dt - \int_{a}^{0}f(t)dt - \int_{0}^{-x}f(t)dt\\
		& = \int_{0}^{x}f(t)dt - \int_{0}^{-x}f(t)dt\\
		& = \int_{0}^{x}f(t)dt - \int_{0}^{x}f(-u)(-du) \hspace{5mm} \text{substituting $\displaystyle{t=-u}$}\\
		& = \int_{0}^{x}f(t)dt + \int_{0}^{x}f(-u)du\\
		& = \int_{0}^{x}f(t)dt + \int_{0}^{x}f(-t)dt \hspace{5mm} \text{dummy variable switch}\\
		& = \int_{0}^{x}\Big[f(t) + f(-t)\Big]dt\\
		\therefore \int_{0}^{x}\Big[f(t) + f(-t)\Big]dt & = 0\\
		\therefore \frac{d}{dx}\left[\int_{0}^{x}\Big[f(t) + f(-t)\Big]dt \right] & = \frac{d}{dx}\left[0\right]\\
		\therefore f(x) + f(-x) & = 0 \hspace{5mm} \text{by the Fundamental Theorem of Calculus}\\
		\therefore f(-x) & = -f(x)\\ 
		\end{align*}

		Therefore $\displaystyle{f}$ is odd if $\displaystyle{A}$ is even.

		\pagebreak

		The second part requires the proof that $\displaystyle{A}$ is even if $\displaystyle{f}$ is odd, which is as follows.

		\begin{align*}
		\int_{-x}^{x}f(t)dt & = 0 \hspace{5mm} \text{as $\displaystyle{f}$ is odd}\\
		\therefore LHS & = \int_{0}^{x}f(t)dt + \int_{-x}^{0}f(t)dt\\
		& = \int_{0}^{x}f(t)dt - \int_{0}^{-x}f(t)dt\\
		& = \int_{0}^{x}f(t)dt + \int_{a}^{0}f(t)dt - \int_{a}^{0}f(t)dt - \int_{0}^{-x}f(t)dt\\
		& = \int_{a}^{x}f(t)dt - \int_{a}^{-x}f(t)dt\\
		\therefore \int_{a}^{x}f(t)dt - \int_{a}^{-x}f(t)dt & = 0\\
		\therefore \int_{a}^{x}f(t)dt & = \int_{a}^{-x}f(t)dt\\
		\therefore A(x) & = A(-x)\\
		\end{align*}

		Therefore $\displaystyle{A}$ is even if $\displaystyle{f}$ is odd. Thus $\displaystyle{A}$ is even if and only if $\displaystyle{f}$ is odd.
		
		\pagebreak

		\item We are now required to prove that $\displaystyle{A}$ is odd if and only if $\displaystyle{f}$ is even and $\displaystyle{A(0)=0}$. In order to complete this proof, we must prove the result from both sides of the implication. For the first part of the proof, we are required to prove $\displaystyle{f}$ is even, and $\displaystyle{A(0)=0}$ if $\displaystyle{A}$ is odd.

		\bigbreak

		Firstly, we shall prove that $\displaystyle{A(0)=0}$ if $\displaystyle{A}$ is odd.

		\begin{align*}
		A(-x) & = -A(x)\\
		\therefore A(0) & = -A(0)\\
		\therefore 2A(0) & = 0\\
		\therefore A(0) & = 0\\
		\end{align*}

		Now we shall complete the remainder of the first proof.

		\begin{align*}
		A(x) & = -A(-x)\\
		\therefore A(x) + A(-x) & = 0\\
		\therefore LHS & = \int_{a}^{x}f(t)dt + \int_{a}^{-x}f(t)dt\\
		& = \int_{a}^{0}f(t)dt + \int_{0}^{x}f(t)dt + \int_{a}^{0}f(t)dt + \int_{0}^{-x}f(t)dt\\
		& = 2\int_{a}^{0}f(t)dt + \int_{0}^{x}f(t)dt + \int_{0}^{-x}f(t)dt\\
		& = 2\int_{a}^{0}f(t)dt + \int_{0}^{x}f(t)dt + \int_{0}^{x}f(-u)(-du) \hspace{5mm} \text{substituting $\displaystyle{t=-u}$}\\
		& = 2\int_{a}^{0}f(t)dt + \int_{0}^{x}f(t)dt - \int_{0}^{x}f(-u)du\\
		& = 2\int_{a}^{0}f(t)dt + \int_{0}^{x}f(t)dt - \int_{0}^{x}f(-t)dt \hspace{5mm} \text{dummy variable switch}\\
		& = 2A(0) + \int_{0}^{x}f(t)dt - \int_{0}^{x}f(-t)dt\\
		& = \int_{0}^{x}f(t)dt - \int_{0}^{x}f(-t)dt\\
		\therefore \int_{0}^{x}f(t)dt - \int_{0}^{x}f(-t)dt & = 0\\
		\therefore \int_{0}^{x}f(t)dt & = \int_{0}^{x}f(-t)dt\\
		\therefore \frac{d}{dx}\left[\int_{0}^{x}f(t)dt \right] & = \frac{d}{dx}\left[\int_{0}^{x}f(-t)dt \right]\\
		\therefore f(x) & = f(-x) \hspace{5mm} \text{by the Fundamental Theorem of Calculus}\\
		\end{align*}

		Therefore $\displaystyle{f}$ is even if $\displaystyle{A}$ is odd and $\displaystyle{A(0)=0}$.

		\pagebreak
		The second part requires the proof that $\displaystyle{A}$ is odd if $\displaystyle{f}$ is even and $\displaystyle{A(0)=0}$, which is as follows.

		\begin{align*}
		\int_{-x}^{x}f(t)dt & = 2\int_{0}^{x}f(t)dt \hspace{5mm} \text{as $\displaystyle{f}$ is even}\\
		\therefore \int_{-x}^{x}f(t)dt - 2\int_{0}^{x}f(t)dt & = 0\\
		\therefore LHS & = \int_{-x}^{x}f(t)dt - 2\int_{0}^{x}f(t)dt\\
		& = \int_{-x}^{x}f(t)dt - 2\int_{0}^{x}f(t)dt - 2A(0) \hspace{5mm} \text{as $\displaystyle{A(0)=0}$}\\
		& = \int_{0}^{x}f(t)dt + \int_{-x}^{0}f(t)dt  - 2\int_{0}^{x}f(t)dt - 2A(0)\\
		& = -\int_{0}^{x}f(t)dt + \int_{-x}^{0}f(t)dt - 2\int_{a}^{0}f(t)dt\\
		& = -\int_{0}^{x}f(t)dt - \int_{a}^{0}f(t)dt - \int_{0}^{-x}f(t)dt - \int_{a}^{0}f(t)dt\\
		& = -\int_{a}^{x}f(t)dt - \int_{a}^{-x}f(t)dt\\
		\therefore -\int_{a}^{x}f(t)dt - \int_{a}^{-x}f(t)dt & = 0\\
		\therefore -\int_{a}^{x}f(t)dt & = \int_{a}^{-x}f(t)dt\\
		\therefore - A(x) & = A(-x)\\
		\end{align*}

		Therefore $\displaystyle{A}$ is odd if $\displaystyle{f}$ is even and $\displaystyle{A(0)=0}$. Therefore $\displaystyle{A}$ is odd if and only if $\displaystyle{f}$ is even and $\displaystyle{A(0)=0}$.

		\end{enumerate}

	\end{enumerate}

\end{enumerate}

\end{document}