% Begin the document and set up the style of the document
\documentclass[a4paper,11pt]{article}

% Install the required packages for the document 
\usepackage{enumitem}
\usepackage{amsmath}
\usepackage{amssymb}
\usepackage{verbatim}
\usepackage{mathtools}

% Page and style settings
%\parskip=8pt
\parindent=0pt
% Right margin
\textwidth=6.25in
% Left margin
\oddsidemargin=0pt
\evensidemargin=0pt
% Bottom margin
\textheight=10in
% Top margin
\topmargin=-0.75in
\baselineskip=11pt
% end of page and other style settings

\renewcommand{\familydefault}{\sfdefault}

\newcommand{\indep}{\mathrel{\text{\scalebox{1.07}{$\perp\mkern-10mu\perp$}}}}
\newcommand{\p}{\mathbb{P}}
\newcommand{\e}{\mathbb{E}}
\newcommand{\ds}{\displaystyle}
\newcommand{\code}{\texttt}
\newcommand{\HRule}{\rule{\linewidth}{0.5mm}} % Defines a new command for the horizontal lines, change thickness here

\newenvironment{nscentre}
 {\parskip=0pt\par\nopagebreak\centering}
 {\par\noindent\ignorespacesafterend}


\usepackage{fullpage}

\usepackage{titlesec} % Used to customize the \section command
\titleformat{\section}{\bf}{}{0em}{}[\titlerule] % Text formatting of sections
\titlespacing*{\section}{0pt}{3pt}{3pt} % Spacing around sections

\begin{document}
\setlength{\abovedisplayskip}{8pt}{%
\setlength{\belowdisplayskip}{0pt}{%


\text{School of Mathematics}
\hfill
\text{University of New South Wales}

\begin{nscentre}
	\textbf{MATH2701: Abstract Algebra and Fundamental Analysis}\\
	\textbf{Short Assignment 1}\\
\end{nscentre}

\text{Name: Keegan Gyoery}
\hfill
\text{zID: z5197058}

\pagenumbering{arabic}

	\begin{enumerate}[leftmargin=*]
		\item Let $\ds{GL_n(\mathbb{R}) \coloneqq \{A \in M_n(\mathbb{R}) \: | \: A \text{ is invertible}\}}$ be the general linear group. Show that
		\begin{align*}
			O_n(\mathbb{R}) & = \{Q \in GL_n(\mathbb{R}) \: | \: Q^{T}Q = I\}\\
		\end{align*}
		is a subgroup.
		\bigbreak
		By the subgroup lemma, we have to show that $\ds{O_n(\mathbb{R})}$ is a non-empty subset of $\ds{GL_n(\mathbb{R})}$, and it satisfies the closure conditions. Let $\ds{M \in O_n(\mathbb{R})}$. By definition of $\ds{O_n(\mathbb{R})}$, $\ds{M \in GL_n(\mathbb{R})}$. Thus, $\ds{O_n(\mathbb{R}) \subseteq GL_n(\mathbb{R})}$. Furthermore, clearly $\ds{I \in O_n(\mathbb{R})}$, so $\ds{O_n(\mathbb{R})}$ is a non-empty subset of $\ds{GL_n(\mathbb{R})}$. Considering closure under composition, let $\ds{M_1, M_2 \in O_n(\mathbb{R})}$, such that
		\begin{align*}
			M_1^TM_1 & = I \dots\dots (A)\\ 
			M_2^TM_2 & = I \dots\dots (B).\\ 
		\end{align*}
		Further, it is clear that $\ds{M_1M_2 \in GL_n(\mathbb{R})}$. Consider now 
		\begin{align*}
			(M_1M_2)^T(M_1M_2) & = (M_2^TM_1^T)(M_1M_2)\\
							   & = M_2^T(M_1^TM_1)M_2\\
							   & = M_2^T(I)M_2 \text{ by } (A)\\
							   & = M_2^TM_2\\
			\therefore (M_1M_2)^T(M_1M_2) & = I \text{ by } (B)\\
		\end{align*}
		and so $\ds{M_1M_2 \in O_n(\mathbb{R})}$, satisfying closure under composition. Examining closure under inverses, let $\ds{M \in O_n(\mathbb{R})}$, we claim $\ds{M^{-1} = M^T}$. As $\ds{M \in GL_n(\mathbb{R})}$, clearly $\ds{M^T \in GL_n(\mathbb{R})}$, and so $\ds{M^{-1} \in GL_n(\mathbb{R})}$. As $\ds{M \in O_n(\mathbb{R})}$, $\ds{M^TM = I \dots\dots (C)}$. Now, consider
		\begin{align*}
			(M^{-1})^T(M^{-1}) & = (M^T)^{-1}(M^{-1})\\
							   & = (MM^T)^{-1}\\
							   & = \big((M^TM)^T\big)^{-1}\\
							   & = \big((I)^T\big){-1} \text{ by } (C)\\
							   & = (I)^{-1}\\
			\therefore (M^{-1})^T(M^{-1}) & = I\\
		\end{align*}
		thus $\ds{M^{-1} = M^T \in O_n(\mathbb{R})}$. It is now clear to see that $\ds{O_n(\mathbb{R})}$ is a subgroup of $\ds{GL_n(\mathbb{R})}$.

		\pagebreak

	\item Let $\ds{\tau : \mathbb{R}^n \rightarrow \mathbb{R}^n}$ be an isometry and assume that $\ds{\tau(\mathbf{0}) = \mathbf{0}}$. Show that
		\begin{enumerate}
			\item $\ds{\tau}$ preserves the dot product on $\ds{\mathbb{R}^n}$: $\ds{\tau(\mathbf{x}) \cdot \tau(\mathbf{y}) = \mathbf{x} \cdot \mathbf{y}}$, for all $\ds{\mathbf{x}, \mathbf{y} \in \mathbb{R}^n}$.
				\bigbreak
				As $\ds{\tau}$ is an isometry, for all $\ds{\mathbf{x}, \mathbf{y} \in \mathbb{R}^n}$, we have the result $\ds{{\Vert \tau(\mathbf{x}) - \tau(\mathbf{y}) \Vert}^2 = {\Vert \mathbf{x} - \mathbf{y} \Vert}^2 \dots (A)}$. Considering the case when $\ds{\mathbf{y} = \mathbf{0}}$, coupled with $\ds{\tau(\mathbf{0}) = \mathbf{0}}$, the above equation yields \\ $\ds{{\Vert \tau(\mathbf{x}) \Vert}^2 = {\Vert \mathbf{x} \Vert}^2 \dots (B)}$. Returning to the equation labelled $\ds{(A)}$, we are able to deduce that $\ds{\tau}$ preserves the dot product, with the aid of equation $\ds{(B)}$.
				\begin{align*}
					{\Vert \tau(\mathbf{x}) - \tau(\mathbf{y}) \Vert}^2 & = {\Vert \mathbf{x} - \mathbf{y} \Vert}^2\\
					\therefore [\tau(\mathbf{x}) - \tau(\mathbf{y})][\tau(\mathbf{x}) - \tau(\mathbf{y})] & = (\mathbf{x} - \mathbf{y})(\mathbf{x} - \mathbf{y})\\
					\therefore \tau(\mathbf{x})\cdot\tau(\mathbf{x}) - 2\tau(\mathbf{x})\cdot\tau(\mathbf{y}) + \tau(\mathbf{y})\cdot\tau(\mathbf{y}) & = \mathbf{x}\cdot\mathbf{x} - 2\mathbf{x}\cdot\mathbf{y} + \mathbf{y}\cdot\mathbf{y}\\
					\therefore {\Vert \tau(\mathbf{x}) \Vert}^2 - 2\tau(\mathbf{x})\cdot\tau(\mathbf{y}) + {\Vert \tau(\mathbf{y}) \Vert}^2 & = {\Vert \mathbf{x} \Vert}^2 - 2\mathbf{x}\cdot\mathbf{y} + {\Vert \mathbf{y} \Vert}^2\\
					\therefore {\Vert \tau(\mathbf{x}) \Vert}^2 - 2\tau(\mathbf{x})\cdot\tau(\mathbf{y}) + {\Vert \tau(\mathbf{y}) \Vert}^2 & = {\Vert \tau(\mathbf{x}) \Vert}^2 - 2\mathbf{x}\cdot\mathbf{y} + {\Vert \tau(\mathbf{y}) \Vert}^2 \text{ by } (B)\\
					\therefore - 2\tau(\mathbf{x})\cdot\tau(\mathbf{y}) & = - 2\mathbf{x}\cdot\mathbf{y}\\
					\therefore \tau(\mathbf{x})\cdot\tau(\mathbf{y}) & = \mathbf{x}\cdot\mathbf{y}\\
				\end{align*}
				
			\item if $\ds{\{\mathbf{e}_1, \mathbf{e}_2, \dots , \mathbf{e}_n\}}$ is the standard basis for $\ds{\mathbb{R}^n}$, then the matrix	
				\begin{align*}
					Q & = (\tau(\mathbf{e}_1), \tau(\mathbf{e}_2), \dots , \tau(\mathbf{e}_n))\\
				\end{align*}
				is orthogonal.
				\bigbreak
				Let $\ds{S = \{\mathbf{e}_1, \mathbf{e}_2, \dots , \mathbf{e}_n\}}$. As $\ds{S}$ is the standard basis, $\ds{S}$ is orthonormal, by definition 
				\begin{align*}
					\mathbf{e}_i \cdot \mathbf{e}_j & = 
					\begin{cases}
						\: 0 \hspace{5mm} \text{if } i \neq j\\
						\: 1 \hspace{5mm} \text{else}\\
					\end{cases} \dots\dots (A).\\
				\end{align*}
				For $\ds{Q}$ to be orthogonal, $\ds{Q^TQ = I}$. This is equivalent to 
				\begin{align*}
					\tau(\mathbf{e}_i) \cdot \tau(\mathbf{e}_j) & = 
					\begin{cases}
						\: 0 \hspace{10mm} \text{if } i \neq j\\
						\: > 0 \hspace{6mm} \text{else}\\
					\end{cases}\\
				\end{align*}
				As we know from the previous question, $\ds{\tau}$ preserves the dot product, and so\\$\ds{\tau(\mathbf{e}_i) \cdot \tau(\mathbf{e}_j) =	\mathbf{e}_i \cdot \mathbf{e}_j}$ for all $\ds{i,j}$. Thus, using result $\ds{(A)}$, we satisfy the above requirements for $\ds{Q}$ to be orthogonal, and thus $\ds{Q = (\tau(\mathbf{e}_1), \tau(\mathbf{e}_2), \dots , \tau(\mathbf{e}_n))}$ is orthogonal.

			\item $\ds{\tau = T_{Q,\mathbf{0}}}$ is a linear isomorphism.
				\bigbreak
				From the Theorem in the Lecture Notes, we can decompose any isometry on $\ds{\mathbb{R}^n}$ into a translation composed with multiplication by an orthogonal matrix. That is, \\$\ds{\tau(\mathbf{x}) = T_{A,\mathbf{b}}(\mathbf{x}) = A\mathbf{x} + \mathbf{b}}$, where $\ds{A}$ is orthogonal, and $\ds{\mathbf{b}}$ is the vector of translation. By the construction of $\ds{Q}$ in the previous part, $\ds{Q(\mathbf{e}_i) = \tau(\mathbf{e}_i)}$, for all $\ds{i}$, and $\ds{Q}$ is orthogonal. Thus, let $\ds{A = Q}$. Furthermore, as $\ds{\tau(\mathbf{0}) = \mathbf{0}}$, we have $\ds{\tau(\mathbf{0}) = T_{Q,\mathbf{b}}(\mathbf{0}) = Q(\mathbf{0}) + \mathbf{b} = \mathbf{b} = \mathbf{0}}$. As $\ds{\mathbf{b} = \mathbf{0}}$, we get the result $\ds{\tau = T_{Q,\mathbf{0}} = Q}$. As $\ds{Q}$ is a linear map, and invertible, $\ds{\tau = T_{Q,\mathbf{0}}}$ is a linear isomorphism. 
				\bigbreak
				\begin{center}
				\emph{This assignment is completely my own work except where acknowledged}\\
				\emph{signed:} \hspace{50mm} \emph{date:}\\
				\end{center}


		\end{enumerate}





	\end{enumerate}
\end{document}
