% Begin the document and set up the style of the document
\documentclass[a4paper,11pt]{article}

% Install the required packages for the document 
\usepackage{enumitem}
\usepackage{amsmath}
\usepackage{amssymb}
\usepackage{verbatim}
\usepackage{mathtools}
\usepackage{tikz}
\usepackage{nicefrac}

% Page and style settings
%\parskip=8pt
\parindent=0pt
% Right margin
\textwidth=6.25in
% Left margin
\oddsidemargin=0pt
\evensidemargin=0pt
% Bottom margin
\textheight=10in
% Top margin
\topmargin=-0.75in
\baselineskip=11pt
% end of page and other style settings

\renewcommand{\familydefault}{\sfdefault}
\usepackage{calrsfs}
\DeclareMathAlphabet{\pazocal}{OMS}{zplm}{m}{n}

\newcommand{\indep}{\mathrel{\text{\scalebox{1.07}{$\perp\mkern-10mu\perp$}}}}
\newcommand{\p}{\mathbb{P}}
\newcommand{\e}{\mathbb{E}}
\newcommand{\ds}{\displaystyle}
\newcommand{\code}{\texttt}
\newcommand{\HRule}{\rule{\linewidth}{0.5mm}} % Defines a new command for the horizontal lines, change thickness here

\newenvironment{nscentre}
 {\parskip=0pt\par\nopagebreak\centering}
 {\par\noindent\ignorespacesafterend}


\usepackage{fullpage}

\usepackage{titlesec} % Used to customize the \section command
\titleformat{\section}{\bf}{}{0em}{}[\titlerule] % Text formatting of sections
\titlespacing*{\section}{0pt}{3pt}{3pt} % Spacing around sections

\begin{document}
\setlength{\abovedisplayskip}{8pt}{%
\setlength{\belowdisplayskip}{8pt}{%


\text{School of Mathematics}
\hfill
\text{University of New South Wales}

\begin{nscentre}
	\textbf{MATH2701: Abstract Algebra and Fundamental Analysis}\\
	\textbf{Short Assignment 2}\\
\end{nscentre}

\text{Name: Keegan Gyoery}
\hfill
\text{zID: z5197058}

\pagenumbering{arabic}
	\begin{enumerate}[leftmargin=*]
		\item 
			\begin{enumerate}
				\item As $\ds{f(x) = O(h(x))}$, for some $\ds{M_1 > 0}$, and some $\ds{x_1}$, we have that for all $\ds{x > x_1}$,
					\begin{align*}
					|f(x)| & \leq M_1 |h(x)|. 
					\end{align*}
					Similarly, as $\ds{g(x) = O(h(x))}$, for some $\ds{M_2 > 0}$, and some $\ds{x_2}$, we have that for all $\ds{x > x_2}$,
					\begin{align*}
						|g(x)| & \leq M_2 |h(x)|.
					\end{align*}
					Now, select $\ds{M = M_1 + M_2}$, and $\ds{x_0 = \max\{x_1, x_2\}}$, such that for all $\ds{x > x_0}$, we have
					\begin{align*}
						|f(x) + g(x)| & \leq |f(x)| + |g(x)|\\
									  & \leq M_1 |h(x)| + M_2 |h(x)|\\
									  & = (M_1 + M_2) |h(x)|\\
						\therefore |f(x) + g(x)| & \leq M |h(x)|.
					\end{align*}
					Thus, $\ds{f(x) + g(x) = O(h(x))}$.
					\bigbreak

				\item As $\ds{f(x) = O(g(x))}$, for some $\ds{M_1 > 0}$, and some $\ds{x_1}$, we have that for all $\ds{x > x_1}$,
					\begin{align*}
					|f(x)| & \leq M_1 |g(x)|. 
					\end{align*}
					Similarly, as $\ds{g(x) = O(h(x))}$, for some $\ds{M_2 > 0}$, and some $\ds{x_2}$, we have that for all $\ds{x > x_2}$,
					\begin{align*}
						|g(x)| & \leq M_2 |h(x)|.
					\end{align*}
					Now, select $\ds{M = M_1M_2}$, and $\ds{x_0 = \max\{x_1, x_2\}}$, such that for all $\ds{x > x_0}$, we have
					\begin{align*}
						|f(x)| & \leq M_1 |g(x)|\\
							   & \leq M_1 (M_2 |h(x)|)\\
									  & = (M_1M_2) |h(x)|\\
						\therefore |f(x)| & \leq M |h(x)|,
					\end{align*}
					Thus, $\ds{f(x) = O(h(x))}$.
					\bigbreak

				\item We have $\ds{f(x) \sim g(x)}$ as $\ds{x \rightarrow a}$, so 
					\begin{align*}
						\lim_{x \to a} \frac{f(x)}{g(x)} & = 1.
					\end{align*}
					Also, $\ds{h(x) = o(g(x)}$ as $\ds{x \rightarrow a}$, so
					\begin{align*}
						\lim_{x \to a} \frac{f(x)}{g(x)} & = 0.
					\end{align*}
					Considering the above two limits, we have
					\begin{align*}
						\lim_{x \to a}\left(\frac{f(x) + h(x)}{g(x)}\right) & = \lim_{x \to a}\left(\frac{f(x)}{g(x)} + \frac{h(x)}{g(x)}\right) = 1 + 0 = 0\\
					\end{align*}
					By definition, $\ds{f(x) + h(x) \sim g(x)}$.
					\bigbreak

				\item Consider $\ds{f(x) = x^3 + x^2 = O(x^4 + x)}$ and $\ds{g(x) = x^3 = O(x^4)}$. This gives $\ds{h(x) = x^4 + x}$ and $\ds{k(x) = x^4}$. Then $\ds{f(x) - g(x) = x^2}$, but $\ds{h(x) - k(x) = x}$. Clearly, $\ds{x^2 \neq O(x)}$, and so by counter-example, the assertion is false.
					\bigbreak
			\end{enumerate}

		\item 
			\begin{enumerate}
				\item Consider the generalised AM-GM Inequality given in lectures, 
					\begin{align*}
						(x_1x_2\dots x_n)^{\nicefrac{1}{n}} & \leq \frac{1}{n} \sum^n_{k=1} x_k,
					\end{align*}
					Noting that equality only occurs when $\ds{x_1 = x_2 = \dots = x_n}$. Let $\ds{x_k = k}$. This gives,
					\begin{align*}
						\sum^n_{k=1} x_k = \frac{n(n+1)}{2}, \hspace{5mm} \text{ and } & \hspace{5mm} x_1x_2\dots x_n = n!.
					\end{align*}
					Using the generalised AM-GM Inequality given above, and the choice of $\ds{x_k}$, 
					\begin{align*}
						\therefore (n!)^{\nicefrac{1}{n}} & \leq \frac{1}{n} \cdot \frac{n(n+1)}{2} = \frac{n+1}{2}\\
						\therefore (n!)^{\nicefrac{1}{n}} & \leq \frac{n+1}{2}\\
						\therefore n! & \leq \left(\frac{n+1}{2}\right)^n.
					\end{align*}
					Note that equality occurs when $\ds{x_1 = x_2 = \dots = x_n}$. From our choice of $\ds{x_k = k}$, the equality condition becomes $\ds{n = n-1 = \dots = 1}$, and thus $\ds{n = 1}$ for equality to occur.
					\bigbreak

				\item Consider the generalised AM-GM Inequality given in lectures, 
					\begin{align*}
						(x_1x_2\dots x_n)^{\nicefrac{1}{n}} & \leq \frac{1}{n} \sum^n_{k=1} x_k.
					\end{align*}
					Applying the inequality to the two factors of the LHS,  
					\begin{align*}
						\left(\sum^n_{k=1} x_k\right) & \geq n(x_1x_2\dots x_n)^{\nicefrac{1}{n}},\\
						\left(\sum^n_{k=1} \frac{1}{x_k}\right) & \geq n\left(\frac{1}{x_1x_2\dots x_n}\right)^{\nicefrac{1}{n}}.
					\end{align*}
					Now, considering the LHS of the result we have to prove, 
					\begin{align*}
						\left(\sum^n_{k=1} x_k\right)\left(\sum^n_{k=1} \frac{1}{x_k}\right) & \geq \left(n(x_1x_2\dots x_n)^{\nicefrac{1}{n}}\right)\left(n\left(\frac{1}{x_1x_2\dots x_n}\right)^{\nicefrac{1}{n}}\right)\\
																							 & = n^2(x_1x_2\dots x_n)^{\nicefrac{1}{n}}\left(\frac{1}{x_1x_2\dots x_n}\right)^{\nicefrac{1}{n}}\\
																							 & = n^2\left(\frac{x_1x_2\dots x_n}{x_1x_2\dots x_n}\right)^{\nicefrac{1}{n}}\\
																							 & = n^2\\
						\therefore \left(\sum^n_{k=1} x_k\right)\left(\sum^n_{k=1} \frac{1}{x_k}\right) & \geq n^2.
					\end{align*}
					\bigbreak
			\end{enumerate}

		\item 
			\begin{enumerate}
				\item Consider first $\ds{e - e_n}$, using their definitions, 
					\begin{align*}
						e - e_n & = \sum^{\infty}_{k=0} \frac{1}{k!} - \sum^{n}_{k=0} \frac{1}{k!}
\\
								& = \sum^{\infty}_{k=n+1} \frac{1}{k!}\\
								& = \frac{1}{(n+1)!} + \frac{1}{(n+2)!} + \frac{1}{(n+3)!} + \dots \\
								& = \frac{1}{(n+1)!}\left(1 + \frac{1}{n+2} + \frac{1}{(n+2)(n+3)} + \dots \right).\\
					\end{align*}
					For all $\ds{k > 1}$, $\ds{n > 0}$, we have $\ds{n + k > n + 1}$, so
					\begin{align*}
						\frac{1}{n+k} & < \frac{1}{n+1}.
					\end{align*}
					Thus, using the result from above,
					\begin{align*}
						e - e_n & = \frac{1}{(n+1)!}\left(1 + \frac{1}{n+2} + \frac{1}{(n+2)(n+3)} + \dots \right)\\
								& < \frac{1}{(n+1)!}\left(1 + \frac{1}{n+1} + \frac{1}{(n+1)^2} + \dots \right)\\
								& = \frac{1}{(n+1)!}\left(\frac{1}{1 - \frac{1}{n+1}}\right)\\
								& = \frac{1}{(n+1)!}\left(\frac{1}{\frac{n+1-1}{n+1}}\right)\\
								& = \frac{1}{(n+1)!}\left(\frac{n+1}{n}\right)\\
								& = \frac{1}{n\cdot n!}\\
						\therefore e - e_n & \leq \frac{1}{n\cdot n!}.
					\end{align*}
					Thus, let $\ds{M = 1}$, and $\ds{n_0 = 1}$, we have
					\begin{align*}
						e - e_n & \leq \frac{1}{n\cdot n!}\\
						\therefore e - e_n & \leq M \left|\frac{1}{n \cdot n!}\right|,
					\end{align*}
					so $\ds{e - e_n = O\left(\frac{1}{n\cdot n!}\right)}$.
					\pagebreak

				\item Clearly, $\ds{e - e_n > 0}$. So, for all $\ds{n > 1}$,
					\begin{align*}
						0 < e - e_n & < \frac{1}{n \cdot n!}.
					\end{align*}
					As $\ds{n > 1}$, then $\ds{\frac{1}{n} < 1}$, so,
					\begin{align*}
						0 & < e - e_n < \frac{1}{n}\cdot \frac{1}{n!}\\
						0 & < e - e_n < \frac{1}{n!}\\
						0 & < n!(e - e_n) < 1.
					\end{align*}
					\bigbreak

				\item Assume $\ds{e \in \mathbb{Q}}$, so there exists co-prime integers $\ds{a}$ and $\ds{b > 0}$, such that $\ds{e = \frac{a}{b}}$. Let $\ds{n = b}$, so $\ds{n \in \mathbb{N}}$. Using the result from the previous part, 
					\begin{align*}
						0 & < n!(e - e_n) < 1\\
						0 & < b!\left(\frac{a}{b} - e_b\right) < 1\\
						0 & < b!\frac{a}{b} - b!\sum^{b}_{k=0} \frac{1}{k!} < 1\\
						0 & < a(b-1)! - \sum^{b}_{k=0} b(b-1)\dots (k+1) < 1.\\
					\end{align*}
					Clearly, $\ds{a(b-1)! \in \mathbb{Z}}$, similarly, $\ds{\sum^{b}_{k=0} b(b-1)\dots (k+1) \in \mathbb{Z}}$. As a result,
					\begin{align*}
						a(b-1)! - \sum^{b}_{k=0} b(b-1)\dots (k+1) & \in \mathbb{Z}. 
					\end{align*}
					However, from the result above,
					\begin{align*}
						a(b-1)! - \sum^{b}_{k=0} b(b-1)\dots (k+1) \in (0,1).
					\end{align*}
					This is a contradiction, and so $\ds{e \notin \mathbb{Q}}$, and thus $\ds{e}$ is irrational.
			\end{enumerate}

	\end{enumerate}
\end{document}
