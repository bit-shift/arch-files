\documentclass[a4paper]{article}

\usepackage{url}
\usepackage{amsmath}
\usepackage{amssymb}
\usepackage{hyperref}

\usepackage{tikz}
\usepackage{graphicx}
\usepackage{enumitem}

\usepackage{titlesec}

\renewcommand{\v}[1]{\textbf{#1}}
\newcommand{\ip}[2]{\langle #1, #2 \rangle}
\newcommand{\e}[1]{\varepsilon_{#1}(u)}
\newcommand{\dote}[1]{\dot{\varepsilon}_{#1}(u)}
\newcommand{\dotg}{\dot{\gamma}(u)}
\newcommand{\ddotg}{\ddot{\gamma}(u)}
\newcommand{\ds}{\displaystyle}
\newcommand{\code}{\texttt}
\newcommand{\HRule}{\rule{\linewidth}{0.5mm}} % Defines a new command for the horizontal lines, change thickness here
\DeclareMathOperator{\spn}{span}

\parskip=8pt
\parindent=0pt
\textwidth=6.25in
\oddsidemargin=0pt
\evensidemargin=0pt
\textheight=10in
\topmargin=-0.75in
\baselineskip=11pt

\begin{document}
\pagenumbering{arabic}
\text{School of Mathematics}
\hfill
\text{University of New South Wales}
\begin{center}
  \vspace{-3mm}
	\textbf{\large {MATH3701: Higher Topology and Differential Geometry \\
	Assignment}}
\end{center}
\vspace{-5mm}
\text{Name: Keegan Gyoery}
\hfill
\text{zID: z5197058}

\begin{enumerate}[leftmargin=*]
\item[\textbf{1.}]
	Consider $\ds{d(l \gamma(u)), u \in I}$. Note that $\ds{l}$ is linear and so $\ds{dl = l}$.
	\begin{align*}
		d(l \circ \gamma(u)) & = d\gamma(u) \cdot dl(\gamma(u)) \\
							 & = d\gamma(u) \cdot l \circ \gamma(u) \hspace{3mm} \text{as $l$ is linear} \\
							 & = d\gamma(u) \cdot c \dots (A).
	\end{align*}
	We know that $\ds{\gamma}$ sends $\ds{u}$ into $\ds{C \subset P}$. For any $\ds{x = \gamma(u)}$ inside the plane, by definition, \\$\ds{l \circ \gamma(u) =l(x) = c}$. Alternatively, we can differentiate as follows.
	\begin{align*}
		\therefore d(l \circ \gamma(u)) & = d(l(x)) \\
										& = d(c) \\
										& = 0 \dots (B)
	\end{align*}
	Using both $\ds{(A)}$ and $\ds{(B)}$, we arrive at the follwing result.
	\begin{align*}
		\therefore d\gamma(u) \cdot c & = 0 \\
		d\gamma(u) & = 0.
	\end{align*}
	Thus all derivatives of $\ds{\gamma}$ are zero, so they cannot be linearly independent in $\ds{\mathbb{R}^4}$. Hence $\ds{\gamma}$ is not Frenet. Geometrically speaking, as the curve's derivatives are not linearly independent and all $\ds{0}$, the curve may not move out of the plane it resides in.
\item[\textbf{2.}]
	\begin{align*}
		\gamma(u) & = (-u,\sin{2u}, \cos{2u})^\mathsf{T} \\
		\dot{\gamma}(u) & = (-1, 2\cos{2u}, -2\sin{2u})^\mathsf{T} \\
		\ddot{\gamma}(u) & = (0, -4\sin{2u}, -4\cos{2u})^\mathsf{T}.
	\end{align*}
	Consider $\ds{\alpha \dotg + \beta \ddotg = \mathbf{0}}$, where $\ds{\alpha,\beta \in \mathbb{R}}$. Solving for $\ds{\alpha}$ and $\ds{\beta}$, we use the following matrix.
	\begin{align*}\left[
	\begin{array}{cc|c}
		-1 & 0 & 0\\
		2\cos2u & -4\sin2u & 0 \\
		-2\sin2u & -4\cos2u & 0 \\
	\end{array}
	\right]
	& \sim \left[
	\begin{array}{cc|c}
		1 & 0 & 0\\
		0 & 1 & 0 \\
		-2\sin2u & -4\cos2u & 0 \\
	\end{array}
	\right]
	\end{align*}
	Clearly, $\ds{\alpha = \beta = 0}$, thus $\ds{\dotg}$ and $\ds{\ddotg}$ are linearly independent, and so $\ds{\gamma}$ is Frenet. For the Frenet frame we get the following results for the $\ds{\varepsilon_i}$.
	\begin{align*}
		\varepsilon_1(u) & = \frac{\dot{\gamma}(u)}{|\dot{\gamma}(u)|} = \frac{1}{\sqrt{5}}(-1, 2\cos{2u}, -2\sin{2u})^\mathsf{T}.
	\end{align*}
	Notice that $\ds{\ip{\varepsilon_1(u)}{\ddot{\gamma}(u)} = 0}$ and so $\ds{\ddot{\gamma}(u)}$ is perpendicular to $\ds{\varepsilon_1(u)}$. Hence for $\ds{\varepsilon_2(u)}$ we simply normalise $\ds{\ddot{\gamma}(u)}$.
	\begin{align*}
	\varepsilon_2(u) = (0, -\sin{2u}, -\cos{2u})^{\mathsf{T}}.
	\end{align*}
	For $\ds{\varepsilon_3(u)}$ we take the cross product of the first two vectors in the distinguished Frenet frame.
	\begin{align*}
		\varepsilon_3(u) & = \varepsilon_1(u) \times \varepsilon_2(u) \\
						 & = \frac{1}{\sqrt{5}}(-1, 2\cos{2u}, -2\sin{2u})^\mathsf{T} \times (0, -\sin{2u}, -\cos{2u})^{\mathsf{T}} \\
						 & = \frac{1}{\sqrt{5}}(-2, -\cos{2u}, \sin{2u})^\mathsf{T}.
	\end{align*}
	Note that $\ds{\det(\varepsilon_1(u) \:\:\: \varepsilon_2(u) \:\:\: \varepsilon_3(u)) = -1}$, so instead choose $\ds{\varepsilon_3(u) = \varepsilon_2(u) \times \varepsilon_1(u) = -\varepsilon_1(u) \times \varepsilon21(u)}$, yielding a determinant of $\ds{+1}$. By the definition of the cross product, and the selections of $\ds{\varepsilon_1(u)}$, $\ds{\varepsilon_2(u)}$, and $\ds{\varepsilon_3(u)}$, they satisfy conditions $\ds{1)}$ and $\ds{2)}$ for a distinguished Frenet frame. Thus the distinguished Frenet frame for $\ds{\gamma}$ is:
	\begin{align*}
		\varepsilon_1(u) & = \frac{1}{\sqrt{5}}(-1, 2\cos{2u}, -2\sin{2u})^\mathsf{T} \\
		\varepsilon_2(u) & = (0, \sin{2u}, \cos{2u})^{\mathsf{T}} \\
		\varepsilon_3(u) &  = \frac{1}{\sqrt{5}}(2, \cos{2u}, -\sin{2u})^\mathsf{T}.
	\end{align*}

\item[\textbf{3.}]
\begin{enumerate}[leftmargin=*]
\item[\textbf{a)}]
	As we have a distinguished Frenet frame for $\ds{\gamma}$, for any $\ds{i, j}$, we have $\ds{\ip{\varepsilon_i(u)}{\varepsilon_j(u)} = 0}$ or $\ds{1}$. In either case we know,
	\begin{align*}
		\frac{d}{du}\ip{\varepsilon_i(u)}{\varepsilon_j(u)} & = 0.
	\end{align*}
	Using the product rule, we get the result,
	\begin{align*}
		\frac{d}{du}\ip{\varepsilon_i(u)}{\varepsilon_j(u)} & = \ip{\dote{i}}{\e{j}} + \ip{\e{i}}{\dote{j}} \\
		\therefore 0 & = \ip{\dote{i}}{\e{j}} + \ip{\e{i}}{\dote{j}} \\
		\therefore \ip{\dote{i}}{\e{j}} & = - \ip{\e{i}}{\dote{j}} \\
		\ip{\dote{i}}{\e{j}} & = - \ip{\dote{j}}{\e{i}} \\
		\therefore w_{ij} & = - w_{ji}.
	\end{align*}
\item[\textbf{b)}]
	Part $\ds{2)}$ of the proposition indicates that for $\ds{1 \leq i \leq m - 1}$, then\\ $\ds{\spn(\e{1}, \cdots , \e{i}) = \spn(\dot{\gamma}(u), \ddot{\gamma}(u), \dots , \gamma^{(i)}(u))}$. Clearly $\ds{\e{i} \in \spn(\e{1}, \cdots , \e{i})}$, \\thus $\ds{\e{i} \in \spn(\dot{\gamma}(u), \ddot{\gamma}(u), \dots , \gamma^{(i)}(u))}$. Hence $\ds{\e{i}}$ can be written as a linear combination,
	\begin{align*}
		\e{i} & = \alpha_1 \dot{\gamma}(u) + \alpha_2 \ddot{\gamma}(u) + \cdots + \alpha_i \gamma^{(i)}(u).
	\end{align*}
	Differentiating yields
	\begin{align*}
		\dote{i} & = \alpha_1 \ddot{\gamma}(u) + \cdots + \alpha_i \gamma^{(i + 1)}(u) \\
		\therefore \dote{i} & \in \spn(\ddot{\gamma}(u), \dots ,\gamma^{(i + 1)}(u)).
	\end{align*}
	As $\ds{\spn(\ddot{\gamma}(u), \dots ,\gamma^{(i + 1)}(u)) \subseteq \spn(\dot{\gamma}(u), \ddot{\gamma}(u), \dots , \gamma^{(i + 1)}(u)) = \spn(\e{1}, \dots , \e{i + 1})}$, then,
	\begin{align*}
		\dote{i} & \in \spn(\dot{\gamma}(u), \ddot{\gamma}(u), \dots , \gamma^{(i + 1)}(u)) \\
		\therefore \dote{i} & \in \spn(\e{1}, \e{2},\dots, \e{i + 1}).
	\end{align*}
	\item[\textbf{c)}]
	Given $\ds{\dote{i} \in \spn(\e{1}, \e{2}, \dots , \e{i + 1})}$ then we can write.
	\begin{align*}
		\dote{i} & = \alpha_1 \e{1} + \alpha_2 \e{2} + \cdots + \alpha_{i + 1} \e{i + 1}. \\
		w_{ij} & = \ip{\dote{i}}{\e{j}} \\
		& = \ip{\alpha_1 \e{1} + \alpha_2 \e{2} + \cdots + \alpha_{i + 1} \e{i + 1}}{\e{j}} \\
		& = \ip{\alpha_1 \e{1}}{\e{j}} + \ip{\alpha_2 \e{2}}{\e{j}} + \cdots + \ip{\alpha_{i + 1} \e{i + 1}}{\e{j}} \\
		& = \begin{cases}
			0 & j > i + 1 \\
			\alpha_j & j \leq i + 1
		\end{cases}
	\end{align*}
	Similarly for $\ds{w_{ji}}$,
	\begin{align*}
		\dote{j} & = \beta_1 \e{1} + \beta_2 \e{2} + \cdots + \beta_{j + 1} \e{j + 1}. \\
		w_{ji} & = \ip{\dote{j}}{\e{i}} \\
		& = \ip{\beta_1 \e{1} + \beta_2 \e{2} + \cdots + \beta_{j + 1} \e{j + 1}}{\e{i}} \\
		& = \ip{\beta_1 \e{1}}{\e{i}} + \ip{\beta_2 \e{2}}{\e{i}} + \cdots + \ip{\beta_{j + 1} \e{j + 1}}{\e{i}} \\
		& = \begin{cases}
			0 & i > j + 1 \\
			\beta_i & i \leq j + 1
		\end{cases}
	\end{align*}
	Furthermore $\ds{\alpha_j > 0}$, and $\ds{\beta_i > 0}$, by positivity of the inner product. We also know that\\ $\ds{w_{ij} = -w_{ji}}$. Hence we have the following cases: \\ [1mm]
	\textbf{1)} $\ds{i > j + 1}$ or $\ds{j > i + 1}$. \\ [1mm]
	In this case either $\ds{w_{ij} = 0}$ or $\ds{w_{ji} = 0}$. Either way, as $\ds{w_{ij} = -w_{ji}}$, we know that $\ds{w_{ij} = 0}$. \\ [1mm]
	\textbf{2)} $\ds{i \leq j + 1}$ and $\ds{j \leq i + 1}$. \\ [1mm]
	In this case, $\ds{i - j \leq 1}$, and $\ds{i - j \geq -1}$. This is equivalent to $\ds{|i - j| = 1}$. \\ [1mm]
	Thus $w_{ij} = 0$ unless $|i - j| = 1$.
	\end{enumerate}

\item[\textbf{4.}]
	Recall we have the following distinguished Frenet frame for $\ds{\gamma}$.
	\begin{align*}
		\varepsilon_1(u) & = \frac{1}{\sqrt{5}}(-1, 2\cos{2u}, -2\sin{2u})^\mathsf{T} \\
		\varepsilon_2(u) & = (0, -\sin{2u}, -\cos{2u})^{\mathsf{T}} \\
		\varepsilon_3(u) &  = \frac{1}{\sqrt{5}}(2, \cos{2u}, -\sin{2u})^\mathsf{T}.
	\end{align*}
	Computing $\ds{\kappa_1(u)}$ we have,
	\begin{align*}
		\kappa_1(u) & = \frac{\ip{\dote{1}}{\e{2}}}{|\dotg|} \\
					& = \frac{\ip{\frac{1}{\sqrt{5}} (0, -4\sin{2u}, -4\cos{2u})^{\mathsf{T}}}{(0, \sin{2u}, \cos{2u})^{\mathsf{T}}}}{\sqrt{5}} \\
					& = \frac{0 + 4\sin^2{2u} + 4\cos^2{2u}}{5} \\
		\therefore \kappa_1(u) & = \frac{4}{5}. 
	\end{align*}
	Computing $\ds{\kappa_2(u)}$ we have,
	\begin{align*}
		\kappa_2(u) & = \frac{\ip{\dote{2}}{\e{3}}}{|\dotg|} \\
					& = \frac{\ip{(0, -2\cos{2u}, 2\sin{2u})^{\mathsf{T}}}{\frac{1}{\sqrt{5}} (2, \cos{2u}, -\sin{2u})^{\mathsf{T}}}}{\sqrt{5}} \\
					& = \frac{0 - 2\sin^2{2u} - 2\cos^2{2u}}{5} \\
		\therefore \kappa_2(u) & = -\frac{2}{5}.
	\end{align*}

\item[\textbf{5.}]
	Given a unit speed Frenet curve $\ds{\gamma : I \rightarrow \mathbb{R}^m}$, we know that $\ds{|\dotg| = 1}$ and, 
	\begin{align*}
		C(u) & = (\dotg, \ddotg, ... , \gamma^{(m)}(u)).
	\end{align*}
	We can also express each column $\ds{\gamma^{(i)}(u)}$ as a linear combination of the first $\ds{i}$ vectors in the Frenet frame.
	\begin{align*}
		\gamma^{(i)}(u) & = \alpha_1 \e{1} + \alpha_2 \e{2} + ... + \alpha_i \e{i}.
	\end{align*}
	


\end{enumerate}
\end{document}
