% Begin the document and set up the style of the document
\documentclass[a4paper]{article}

% Install the required packages for the document 
\usepackage{envmath}
\usepackage{esvect}
\usepackage{graphicx}
\usepackage{gensymb}
\usepackage{tikz}
\usepackage[mathcal]{euscript}
\usepackage{geometry}
\usepackage{enumitem}
\usepackage{mathtools}
\usepackage{subdepth}
\usepackage{graphicx}
\usepackage{amsmath}
\usepackage{amscd}
\usepackage{amssymb}
\usepackage{amsfonts}
\usepackage{harpoon}
\usepackage[title]{appendix}
\usepackage{pgf}
\usepackage{tikz}
\usepackage{mathrsfs}
\usepackage{asyalign}
\usepackage{physics}
\usepackage{enumitem}
\usepackage{xhfill}
\usepackage{accents}
\usepackage{cite}
\usepackage{url}
\usepackage{csquotes}
\usepackage{wrapfig}
\usepackage{booktabs}
\usepackage{adjustbox}
\usepackage{caption}
\usepackage{minipage-marginpar}
\usepackage{calc}
\usepackage[tableposition=top]{caption}
\usepackage{ifthen}
\usepackage[utf8]{inputenc}
\usepackage{tikz-3dplot}
\usetikzlibrary{patterns}
\usetikzlibrary{arrows}

% Page and style settings
\parskip=8pt
\parindent=0pt
% Right margin
\textwidth=6.25in
% Left margin
\oddsidemargin=0pt
\evensidemargin=0pt
% Bottom margin
\textheight=10in
% Top margin
\topmargin=-0.75in
\baselineskip=11pt
% end of page and other style settings

\renewcommand{\familydefault}{\sfdefault}

\newcommand{\indep}{\mathrel{\text{\scalebox{1.07}{$\perp\mkern-10mu\perp$}}}}
\newcommand{\p}{\mathbb{P}}
\newcommand{\e}{\mathbb{E}}
\newcommand{\ds}{\displaystyle}
\newcommand{\code}{\texttt}

% Begin the text of the document
\begin{document}

\newlength{\strutheight}
\settoheight{\strutheight}{\strut}

% Begin the Title Page
\begin{titlepage}

\newcommand{\HRule}{\rule{\linewidth}{0.5mm}} % Defines a new command for the horizontal lines, change thickness here

\center % Center everything on the page
 
\textsc{\LARGE University of New South Wales}\\[1.5cm] % Name of your university/college
\textsc{\Large SCIF 1131}\\[0.5cm] % Major heading such as course name
\textsc{\large Science: Perspectives}\\[0.5cm] % Minor heading such as course title

\HRule \\[0.4cm]
{ \huge \bfseries Interview Contract}\\[0.4cm] % Title of your document
\HRule \\[1.5cm]


\begin{center} \large
Robert Armour (z5217987), Jack English (z5208352)\\ 
Keegan Gyoery (z5197058), Zac Sanchez (z5194994)% Your name
\\
\end{center}


\vspace{4cm}

{\today}\\[3cm] % Date, change the \today to a set date if you want to be precise

\vfill % Fill the rest of the page with whitespace

\end{titlepage}

\pagenumbering{arabic}

\section{Division of Tasks}
\textbf{Rob} - Writing the script.\\
\textbf{Jack} - Filming the interview.\\
\textbf{Keegan} - Interviewing the scientist.\\
\textbf{Zac} - Editing the film of the interview.\\
These are big picture outlines and guides for each team member, as each member will assist in the other tasks where and when needed in order to make it a fluent team project and allow us to complete the task in the most effective manner possible.\\

\section{Deadlines}
Emailing Dr Zika to plan a time will be completed by this weekend, that is the 9th of September.\\
The script will be finished during week , with any final editing occurring during the last few days in week 8, in order to have it prepared before the interview. Ideally each member of the team will add a few questions of their own choosing to the script.\\
The interview will be conducted either at the end of week 8, or week 9, the day is yet to be worked out, and is heavily dependent on Dr Zika's schedule.\\
The editing process will begin once the interview has been conducted, and will be finished by the end of week 10, when the video is due.\\
Further details will be worked out during the process, and will provide a more in-depth and specific outline as to the tasks that must be completed and when by.\\

\section{Meeting Times}
A regular meeting will be conducted before the SCIF tutorial on Wednesday morning to assess progress and ensure that we are on track to complete the required tasks for each week. The time is yet to be confirmed but will most likely be 10am on Wednesday mornings, most likely in the library.\\
If any additional meetings are required, they will be scheduled on a case by case basis.\\

\section{Scientist}
Doctor Jan Zika, Lecturer in Applied Mathematics at the school of Mathematics and Statistics.


\end{document}